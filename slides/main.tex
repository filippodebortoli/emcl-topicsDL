% arara: xelatex
% arara: xelatex
%\documentclass[handout,smaller]{beamer}
\documentclass[smaller, dvipsnames]{beamer}
  \usetheme[numbering=fraction,%
       block=fill,%
       ]{metropolis} % Use metropolis theme
\setbeamercovered{transparent}

\usepackage{tcolorbox}
\tcbuselibrary{theorems}
\usepackage{mathspec}
\setmathfont(Digits,Latin){Fira Sans Light}

\usepackage{microtype}
\usepackage{xcolor,xspace}
\usepackage{notation}
\usepackage{textcomp}
\usepackage{fontawesome}

\newcounter{simprop}
\newcommand{\propstop}{\setcounter{simprop}{\theenumi}}
\newcommand{\propplay}{\setcounter{enumi}{\thesimprop}}

\title{A Framework for Semantic-based Similarity Measures for \elh-Concepts}
\author{Filippo De Bortoli}
\institute{European Master's Program in Computational Logic, TU Dresden}
\date{January 18th, 2018}

\begin{document}
\maketitle

\begin{frame}
  \frametitle{Covered paper}
  Assigned paper:
  \begin{thebibliography}
    \bibitem{}
     Lehmann, K. \& Turhan, A.
	   \newblock{\em A Framework for Semantic-Based Similarity Measures for \elh-Concepts}.
	   \newblock{\bf Logics in Artificial Intelligence - 13th European Conference}, {JELIA} , 2012.
  \end{thebibliography}
\end{frame}

\begin{frame}
  \frametitle{Concept similarity measures (\csm)}
  \begin{itemize}
    \item Functions that assess similarity between concept descriptions
    \begin{itemize}
      \item From 0 (totally dissimilar) to 1 (similar)
    \end{itemize}
    \item Several application areas:
    \begin{itemize}
      \item Bioinformatics: Gene Ontology
      \note{Bioinformatics: Semantic similarity measures employed to compare genes and proteins wrt to their functions, rather than their sequence similarity.}
      \item Geoinformatics: \textsc{sim-dl}
      \note{Geoinformatics: Similarity measures used to improving information retrieval and integration of heterogeneous spatial data sources.}
      \item Ontology alignment
      \note{Ontology alignment: similarity measures used to determine corrispondences between concepts in different ontologies.}
    \end{itemize}
    \item Different kinds of \csm{}s:
    \begin{itemize}
      \item \alert{Structural} measures: \textsc{sim-dl}, adaptations of Jaccard Index, \ldots
      \item \alert{Interpretation-based} measures: Semantic Similarity measure, \ldots
    \end{itemize}
  \end{itemize}
\end{frame}

\begin{frame}
  \frametitle{Motivation for a new concept similarity measure}
  \begin{columns}
    \begin{column}{0.55\textwidth}
      \begin{enumerate}
        \item Are \(Mother\) and \(\text{\faFemale}\, \sqcap \exists{}parentOf.\text{\faUser}\) similar?
        \item Are \(Mother\) and \(Father\) as similar
        as \(\text{\faFemale}\, \sqcap Parent\) and \(Father\) are?
      \end{enumerate}
    \end{column}
    \begin{column}{0.45\textwidth}
        \begin{align*}
          \text{\faFemale}\, &\subsume{} \text{\faUser} \\
          \text{\faMale}\, &\subsume{} \text{\faUser} \\
          Parent &\equiv \exists{}parentOf.\text{\faUser} \\
          Mother &\equiv \text{\faFemale}\, \sqcap Parent \\
          Father &\equiv \text{\faMale}\, \sqcap Parent \\
          &\vdots
        \end{align*}
    \end{column}
  \end{columns}
  \onslide<+->
  \begin{itemize}[<+->]
    \item Equivalence \alert{invariance} and \alert{closure}
    of concept similarity measures (\csm)
    recognised as important properties.
    \item Many existing \csm{}s do \alert{not} satisfy these properties.
  \end{itemize}
\end{frame}

\begin{frame}
  \frametitle{Outline}
  \begin{enumerate}
    \item Preliminaries
    \begin{enumerate}
      \item The \elh Description Logic
      \item \elh-normal form
    \end{enumerate}
    \item Similarity in Description Logic
    \begin{enumerate}
      \item Properties of similarity measures
    \end{enumerate}
    \item The concept similarity measure \(\simi\)
    \begin{enumerate}
      \item Derivation of \(\simi\)
      \item Definition of \(\simi\)
      \item Properties of \(\simi\)
    \end{enumerate}
  \end{enumerate}
\end{frame}

\section{Preliminaries}

\begin{frame}
  \frametitle{The \elh{} Description Logic}
  \begin{block}{\elh}
    \begin{itemize}
      \item Concept names \(\Conc\), role names \(\R\), \elh-concepts \(\celh\)
      \item \alert{Concept constructors}: \(C ::= \top \mid A \mid C \sqcap D \mid \exists{}r.C \quad A \in \Conc\)
      \item (primitive) \alert{concept definitions}: \(A \subsume{} C\), \(A \equiv C\) in a \alert{TBox} \(\mathcal{T}\)
      %\item general concept inclusions: \(C \subsume{} D\) in \(\mathcal{T}\) as well
      \item \alert{role inclusion axioms}: \(r \subsume{} s\) in a \alert{RBox} \(\mathcal{R}\)
    \end{itemize}
  \end{block}
  %\begin{columns}
  %  \begin{column}{0.5\textwidth}
  %    \el concept descriptions
  %    \begin{itemize}[<+->]
  %      \item \(\Conc\) \alert{concept names}, \(\R\) \alert{roles}
  %      \item 
  %      \item \alert{TBox} a finite set of axioms \(A \subsume{} C\), \(C \equiv D\)
  %      \item \(\I \models C \subsume{} D\) iff \(C^\I \subseteq D^\I\)
  %    \end{itemize}
  %  \end{column}
  %  \begin{column}{0.5\textwidth}
  %    \elh = \el + \alert{role hierarchy}
  %    \begin{itemize}[<+->]
  %      \item \alert{Role inclusion axioms} (\ria{}s) \(r \sqsubseteq s\)
  %      \item \alert{RBox} as a finite set of \ria{}s
  %      \item Given an interpretation \(\I\), \(\I \models r \sqsubseteq s\) iff \(r^\I \subseteq s^\I\)
  %    \end{itemize}
  %  \end{column}
  %\end{columns}
  \onslide<+->
  \textbf{Knowledge base}: \(\mathcal{K} = (\mathcal{T}, \mathcal{R})\), with \(\mathcal{T}\) a TBox, \(\mathcal{R}\) an RBox.
  
  \onslide<+->
  \textbf{Assumption:} The TBox is \alert{unfoldable},
  i.e. \textbf{acyclic} and:
  \begin{itemize}
    \item Only contains (primitive) concept definitions
    \item Each concept name at most once in LHS of a definition.
  \end{itemize}
  Unfoldable TBoxes can be \alert{normalized} and input concepts can be expanded, making it possible to work without the TBox.
\end{frame}

\begin{frame}
  \frametitle{Concepts, atoms and normal form}
  For \(C = \bigsqcap_{i \le n} C_i \in \celh\),
  the set of its \alert{atoms} is
  \(
    \atom{C} := \lbrace C_1,\dotsc,C_n \rbrace
  \).
  \note{Intuitively, atoms are concepts not in
  conjunction form.}
  \onslide<+->
  \elh-concepts admit a unique (mod. commutativity) \alert{normal form}, obtained by applying exhaustively:
  \begin{itemize}
    \item \(A \sqcap \top \mapsto A\)
    \item \(A \sqcap A \mapsto A\)
    \item \(\exists{}r.C^\prime \mapsto%
           \exists{}\textcolor{purple}{ch}([r]).C^\prime\),
           where \(\textcolor{purple}{ch}\) is a role choice function
    \item \(\exists{}r.C^\prime \sqcap%
           \exists{}s.D^\prime \mapsto%
          \exists{}r.C^\prime\) if
          \(r\subsume{K}s\) and
          \(C^\prime \subsume{K} D^\prime\)
  \end{itemize}
  \onslide<+->
  \elh-NF incorporates the knowledge from \(\mathcal{R}\) and ensures equivalence \alert{invariance}, since equivalent concepts have the same normal form.
\end{frame}


\section{Similarity measures}

\begin{frame}
  \frametitle{Metric spaces and similarity}
  \begin{enumerate}[<+->]
    \item \alert{Metric spaces} arise in many application areas
    \begin{itemize}
      \item \textsc{Gis}, computer vision, biology,\ldots
    \end{itemize}
    \item Metrics over normalized spaces as \alert{dissimilarity} measures!
    \begin{itemize}
      \item \(d(A,B) = 0\) iff \(A = B\)
      \item Wanted: \(s(A,B) = 1\) iff \(A \equiv B\)
    \end{itemize}
    \item Given metric \(d\) over a normalized space,
    the corresponding \alert{similarity} measure is
    \(s(a,b) := 1 - d(a,b)\).
  \end{enumerate}  
\end{frame}

\begin{frame}
  \frametitle{Properties of \csm{}s}
  Desired properties for a 
  \alert{similarity measure} \(\sme\) (\(C,D,E \in \celh\)):
  \begin{enumerate}[<+->]
    \item \alert{Symmetry}: \(\sme(C,D) = \sme(D,C)\)
    \item \alert{Triangle inequality}:
      \(1 + \sme(D,E) \ge \sme(D,C) + \sme(C,E)\)
    \item\label{equ:1} \alert{Equivalence closed}:
      \(\sme(C,D) = 1\) iff \(C \equiv D\)
    \item\label{equ:2} \alert{Equivalence invariant}:
      if \(C \equiv D\), then
      \(\sme(C,E) = \sme(D,E)\)
    \propstop
  \end{enumerate}
  \onslide<+->
  Properties~\ref{equ:1} and~\ref{equ:2} are useful:
  \begin{itemize}
    \item Equivalent concepts should be similar (property~\ref{equ:1})
    \item They should similarly relate to other concepts (property~\ref{equ:2})
  \end{itemize}
\end{frame}

\begin{frame}
  \frametitle{Properties of similarity, ctd.}
  New properties are introduced:
  \begin{enumerate}[<+->]
    \propplay
    \item \alert{\(\subsume{}\)-preserving}: if \(C \subsume{} D \subsume{} E\),
    then \(\sme(C,D) \ge \sme(C,E)\)
    \item \alert{Reverse \(\sqsubseteq\)-preserving}: 
    \(\sme(C,E) \le \sme(D,E)\) if
    \(C \subsume{} D \subsume{} E\)
    \item \alert{Structurally dependent}: 
    Let \({(C_n)}_n \subseteq \A\) s.t. for \(i \ne j\),
    \(C_i \not\sqsubseteq C_j\).
    
    Then, the concepts
    \(D_n\), \(E_n\) satisfy
    \(\lim_{n \to \infty} \sme(D_n,E_n) = 1\)
    \begin{equation*}
      D_n := \bigsqcap_{i \le n} C_i \sqcap D \qquad
      E_n := \bigsqcap_{i \le n} C_i \sqcap E
    \end{equation*}
    Intuitively, the more \alert{features}
    two concepts share, the more similar they are.
  \end{enumerate}
\end{frame}

\section{Deriving simi}

\begin{frame}
  \frametitle{An intuition for simi}
  \textbf{Recall:} \(C \equiv D\) iff \(C \subsume{} D\) and \(D \subsume{} C\).

  \onslide<+->  
  \alert{Idea} for the definition process:
  \begin{enumerate}[<+->]
    \item Define a measure \(\simi_d\) of \alert{directed similarity}
    \item Ensure that \(\simi_d(C,D) = 1\) iff \(D \subsume{} C\)
    \item Combine \(\simi_d(C,D)\) and \(\simi_d(D,C)\) in a suitable way
  \end{enumerate}
  \onslide<+->
  \begin{definition}[Fuzzy connector]
    Commutative operator
    \(\otimes \colon%
    {[0,1]}^2 \to [0,1]\) s.t. for all
    \(x, y \in [0,1]\),
    \begin{enumerate}%[<+->]
      \item\label{fu:1} \alert{Equivalence closed}:
      \(x \otimes y = 1\) iff \(x = y = 1\)
      \item\label{fu:2} \alert{Weak monotonic}:
      if \(x \le y\), then
      \(1 \otimes x \le 1 \otimes y\) 
      \item\label{fu:3} \alert{Bounded}:
      if \(x \otimes y = 0\), then
      \(x = 0\) or \(y = 0\)
      \item\label{fu:4} \alert{Grounded}:
      \(0 \otimes 0 = 0\)
    \end{enumerate}
  \end{definition}
\end{frame}

\begin{frame}
  \frametitle{simi and equivalence closure}
  \begin{equation}
    \label{eqn:simi}
    \tcboxmath{\simi(C,D) := \simi_d(C,D) \otimes \simi_d(D,C)}
  \end{equation}
  \onslide<+->
  \begin{itemize}[<+->]
    \item Commutativity of \(\otimes\) \textrightarrow{} \alert{symmetry} of \(\simi\)
    \item \(\simi(C,D) = 1\) iff
          \(\simi_d(C,D) = \simi_d(D,C) = 1\)
    \item Thus, \(\simi(C,D) = 1\) iff \(C \subsume{} D\) and \(D \subsume{} C\)
    \item As a consequence of~\eqref{eqn:simi} \(\simi\) is \alert{equivalence closed}! 
  \end{itemize}
\end{frame}

\begin{frame}
  \frametitle{Starting point for the derivation of \(\simi_d\): a set measure}
  First, introduce a variation of the \alert{Jaccard Index}:
  \begin{equation}
    d(C,D) := \frac{\abs{\atom{C} \cap \atom{D}}}%
                   {\abs{\atom{C}}}
  \end{equation}
  \onslide<+->
  What about the numerator?
  \begin{equation}
    \abs{\atom{C} \cap \atom{D}} =
    \sum_{C^\prime \in \atom{C}}%
      \textcolor{cyan}{\max_{D^\prime \in \atom{D}}}%
      (\mydelta_{C^\prime}^{D^\prime})
    \qquad \mydelta_{C^\prime}^{D^\prime}
    = \begin{cases}
      1 & C^\prime = D^\prime \\ 0 & C^\prime \ne D^\prime
    \end{cases}
  \end{equation}
  \onslide<+->
  The function \(\mydelta\) can be improved:
  %\onslide<+->
  \begin{enumerate}[<+->]
    \item it treats different concept names as totally dissimilar 
    \item it ignores the structure of the atoms
    %\item Does not take into account role names
    %\item Cannot properly treat existential restrictions
  \end{enumerate}
\end{frame}

\begin{frame}
  \frametitle{Generalizing \(\delta\) to a primitive measure for names}
  \begin{definition}[Primitive measure]
    Mapping \(\mypm \colon {(\Conc \cup \R)}^2 \to [0,1]\)
    satisfying:
    \begin{enumerate}
      \item\label{pm:1} \(\mypm(A,B) = 1\) iff \(A = B\), where \(A,B \in \Conc\);
      \item\label{pm:2} \(\mypm(r,s) = 1\) iff \(s \subsume{K} r\),
      \item\label{pm:3} if \(r \subsume{K} s\), then \(\mypm(r,s) > 0\),
      \item\label{pm:4} if \(t \subsume{K} s\), then \(\mypm(r,s) \le \mypm(r,t)\), for \(r,s,t \in \R\).
    \end{enumerate}
  \end{definition}
  \onslide<+->
  \begin{itemize}%[<+->]
    \item In conditions~\ref{pm:1} and~\ref{pm:2}, step toward \alert{equivalence closure}
    %\item Rule~\ref{pm:4} needed to ensure
    %\(\subsume{}\)-preserving
  \end{itemize}
\end{frame}

\begin{frame}
  \frametitle{Handling existential restrictions}
  Case analysis (for all \(C,D \in \celh\), \(r,s \in \R\)):
  \begin{enumerate}
    \item \(d(A,B) = \mypm(A,B)\) for \(A,B \in \Conc\)
    \item \(d(\exists{}r.C,D) = d(D,\exists{}r.C) = 0\)
    \item \(d(\exists{}r.C,\exists{}s.D) =
    \mypm(r,s) \cdot \left(\myw + (1-\myw) \cdot d(C,D)\right)\)
  \end{enumerate}
  %\onslide<+->
  \(\myw \in (0,1)\): chooses if roles or concepts accounts more.

  \pause
  In \(\simi_d\), the cases where only \alert{atoms} are defined as:
  \begin{tcolorbox}
    \begin{enumerate}
      \item \(\simi_d(A,B) := \mypm(A,B)\)
      \item \(\simi_d(\exists{}r.C,D) = \simi_d(D,\exists{}r.C) = 0\) for \(D \ne \top\)
      \item \(\simi_d(\exists{}r.C,\exists{}s.D) =
      \mypm(r,s) \cdot \left(\myw + (1-\myw) \cdot \simi_d(C,D)\right)\)
    \end{enumerate}
  \end{tcolorbox}
\end{frame}

\begin{frame}
  \frametitle{Using more knowledge}
  Assume \(\mypm(A,B) = \tfrac{1}{2}\) for all \(A,B \in \Conc\).
  How similar are \(\colour{green}\) and \(\colour{blue} \sqcap \colour{yellow}\)?
     \begin{equation*}
      d(C,D) = \left(\textstyle{\sum_{C^\prime \in \atom{C}}%
      \textcolor{cyan}{\max_{D^\prime \in \atom{D}}}}%
      (\mypm(C^\prime, D^\prime)\right)/(\abs{\atom{C}})
     \end{equation*}
      \begin{equation*}
          \simi(\colour{green}\,\text{\textbf{,}}\,\colour{blue} \sqcap \colour{yellow}) = d(\colour{green}\,\text{\textbf{,}}\,\colour{blue} \sqcap \colour{yellow}) \otimes_{avg} d(\colour{blue} \sqcap \colour{yellow}\,\text{\textbf{,}}\, \colour{green})= \tfrac12 \otimes_{avg} \frac{\tfrac12 + \tfrac12}{2} = \frac{1}{2}
        %\frac{\textcolor{cyan}{\max}(\mypm(\colour{green},\colour{blue}),\mypm(\colour{green},\colour{yellow}))}{1} = \frac{1}{2}
      \end{equation*}

  \(\textcolor{cyan}{\max}\) picks best match, discarding other information on similarity.
  \pause
  \begin{definition}[Bounded t-conorm]
    Operator \(\myconorm \colon {[0,1]}^2 \to [0,1]\) which is
    commutative, associative and for all
    \(x, y, z, w \in [0,1]\):
    \begin{enumerate}
      \item \alert{Unit element}: \(0 \myconorm x = x = x \myconorm 0\) (dissimilar are ininfluent)
      \item \alert{Monotonicity}: if \(x \le z\) and \(y \le w\), then \(x \myconorm y \le z \myconorm w\)
      \item \alert{Boundedness}: if \(x \myconorm y = 1\), then \(x = 1\) or \(y = 1\). (\(\ge 1\) similar)
    \end{enumerate}
  \end{definition}
\end{frame}

\begin{frame}
  \frametitle{Which concepts are more relevant?}
  Some atoms \textit{may be} more important than others
  \begin{itemize}
    \item Contextualization
    \item Priority assignment
  \end{itemize}
  \begin{definition}[Weighting function]
    A \emph{weighting function} is a mapping 
    \(\myg \colon \mathbf{A} \to \mathbb{R}_+\),
     where \(\mathbf{A}\) is the set of \emph{atoms}.
  \end{definition}
  %With this definition, it is possible to partially
  %ensure structural dependence for the similarity
  %measure.
\end{frame}

\section{Definition of simi}

\begin{frame}
  \frametitle{Directed similarity}
  %\alert{Directed similarity} as a measure of subsumption:
  %if \(\sme_d\) measures directed similarity and \(\sme_d(C,D) = 1\), then \(D \subsume{} C\).

  %Introducing a measure of directed similarity:
  \begin{equation}
  \tcboxmath{\simi_d \colon \celh^2 \to [0,1]}
  \end{equation}
  How does \(\simi_d\) behave?
  \pause
  \begin{equation}
    \simi_d(\top,D) = 1 \qquad D \in \celh
  \end{equation}
  Thus, \(D \sqsubseteq \top\) for all \(D \in \celh\).
  \pause
  \begin{equation}
    \simi_d(A,B) = \mypm(A,B) \qquad A,B \in \Conc
  \end{equation}
  \begin{equation}
    \simi_d(\exists{}r.C,\exists{}s.D) =
    \mypm(r,s) \cdot 
    \left(\myw + 
    (1-\myw)\simi_d(C,D)\right)
  \end{equation}
\end{frame}

\begin{frame}
  \frametitle{Directed similarity, ctd.}
  The general case, for \(C,D \ne \top\), is
  \begin{equation}
    \simi_d(C,D) =
    \frac{\sum_{C^\prime \in \atom{C}}\left(\myg(C^\prime) \cdot \textcolor{cyan}{\bigoplus_{D^\prime \in \atom{D}}} \simi_d(C^\prime, D^\prime)\right)}%
    {\sum_{C^\prime \in \atom{C}}\myg(C^\prime)}
  \end{equation}
  Otherwise, \(\simi_d(C,D) = 0\).
  \pause
  \begin{lemma}[Properties of \(\simi_d\)]
    Let \(C,D,E \in \celh\).
    \begin{itemize}
      \item \(\simi_d(C,D) = 1\) iff \(D \subsume{} C\),
      \item if \(D \subsume{} E\), then \(\simi_d(C,D) \le \simi_d(C,E)\).
    \end{itemize}
  \end{lemma}
\end{frame}

\begin{frame}
  \frametitle{Properties of simi}
  \begin{alertblock}{Properties of \(\simi\)}
    The function \(\simi\) is
    \begin{itemize}
      \item symmetric
      \item equivalence invariant
      \item equivalence closed
      \item \(\subsume{}\)-preserving
    \end{itemize}
    for all choices of its parameters.
  \end{alertblock}
  \pause
  \begin{theorem}[Computation of \(\simi\)]
    \(\simi\) can be computed in time
    polynomial in the size of measured concepts,
    if each parameter can be computed in polynomial time.
  \end{theorem}
  \pause
  \(\simi\) does not satisfy the triangle inequality and reverse \(\subsume{}\)-preserving.
\end{frame}

\begin{frame}
  \frametitle{Summary}
  \begin{itemize}[<+->]
    \item Many existing \csm don't satisfy equivalence invariance and closure
    \item \(\simi\) has been introduced
    \begin{itemize}
      \item it satisfies both equivalence closure and invariance, together with other properties
      \item it is \alert{parametric}, thus flexible
      \item it is computable in polynomial time\(^\star\), thus it is \alert{efficient}
    \end{itemize}
  \end{itemize}
  \onslide<+->
  \scriptsize{\(\star\): if the parameters can be computed efficiently.}
\end{frame}

\begin{frame}[standout]
  Thank you!
\end{frame}

\appendix

\begin{frame}
  \frametitle{A counterexample for triangle inequality}
  Let \(G\) and \(F_i\), \(0 \le i \le 4\) be atoms such that for \(i \ne j\), \(\simi_d(F_i,F_j) = \tfrac34\);
  let \(\simi_d(G,F_i) = \simi_d(F_i,G) = 0\).
  \begin{equation*}
    C := G \sqcap \bigsqcap_{i=0}^5F_i
    \quad
    D := G \sqcap F_0
    \quad
    E := G
  \end{equation*}
  In this case, \(C \subsume{} D \subsume{} E\).
  Using Hamacher product as \(\otimes\), \(\mypm_{def}\) and \(\myg_{def}\) as parameters,
  \begin{equation*}
    \simi(C,E) = \frac{1}{6} \quad
    \simi(D,E) = \frac{1}{2} \quad
    \simi(C,D) = \frac{5}{6}
  \end{equation*}
  Consequently, \(1 + \tfrac16 \not\ge \tfrac12 + \tfrac78\).

  Triangle inequality, in general, is not fulfilled by \(\simi\).
\end{frame}
\end{document}
