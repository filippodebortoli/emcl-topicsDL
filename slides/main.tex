\documentclass{beamer}

\usepackage{xcolor}
\usepackage{notation}

\usetheme[numbering=fraction,%
       block=fill%
       ]{metropolis} % Use metropolis theme
\usecolortheme{spruce}
\title{A Framework for Semantic-based Similarity Measures for ELH-Concepts}
\author{Filippo De Bortoli}
\institute{European Master's Programme in Computational Logic, TU Dresden}
\date{January 18th, 2018}
\begin{document}
\maketitle

\begin{frame}
  \frametitle{Similarity measures and the real world}
  \begin{enumerate}
    \item \emph{KB exploration}, e.g. discovery of genetic functionality;
    \item \emph{Ontology alignment}, i.e. how to match different ontologies.
  \end{enumerate}
\end{frame}

\begin{frame}
  \frametitle{Overview}
  \begin{enumerate}
    \item Some preliminaries
    \item Similarity measures and intended properties
    \item Conception of \(\simi\)
    \item Conclusion
  \end{enumerate}
\end{frame}

\section{Preliminaries}

\begin{frame}
  \frametitle{The \elh{} Description Logic}
\end{frame}

\section{First Section}
\begin{frame}{First Frame}
Hello, world!
\end{frame}

\begin{frame}{Preprocessing}
  % EL-normal form here.
\end{frame}

\begin{frame}{Definition of \(\simi_d\)}
  \begin{align}
    \simi_d(\top,D) &= 1 \quad
    \text{for all}\; D \in \mathcal{C}(\mathcal{ELH}) \\
    \simi_d(C,D) &=
    \frac{\sum_{C^\prime \in \widehat{C}}\left(\textcolor{blue}{g}(C^\prime) \cdot \textcolor{green}{\bigoplus_{D^\prime \in \widehat{D}}} \simi_d(C^\prime, D^\prime)\right)}%
    {\sum_{C^\prime \in \widehat{C}}\textcolor{blue}{g}(C^\prime)} \quad \text{if}\; C \ne \top \label{simi:general}\\
    \simi_d(A,B) &= \textcolor{red}{pm}(A,B) \quad \text{for atoms}\; A,B \\
    \simi_d(\exists{}r.C,\exists{}s.D) &=
    \textcolor{red}{pm}(r,s) \cdot \left(\textcolor{orange}{w} + (1-\textcolor{orange}{w})\simi_d(C,D)\right) \\
    \simi_d(C,D) &= 0 \quad \text{otherwise}.
  \end{align}
  % \begin{definition}[\(\simi_d\)]\label{simi-d}
  %   The measure of \emph{directed similarity} \(\simi_d \colon {\mathcal{C}(\mathcal{ELH})}^2 \to [0,1]\), given the following parameters:
  %   \begin{enumerate}
  %     \item a bounded \(t\)-conorm \(\oplus\),
  %     \item a primitive measure \(pm\),
  %     \item a weighting function \(g\),
  %     \item \(w \in (0,1)\),
  %   \end{enumerate}
  %   is defined as:
  %   \begin{align}
  %     \simi_d(\top,D) &= 1 \quad
  %     \text{for all}\; D \in \mathcal{C}(\mathcal{ELH}) \\
  %     \simi_d(C,D) &=
  %     \frac{\sum_{C^\prime \in \widehat{C}}\left(g(C^\prime) \cdot \bigoplus_{D^\prime \in \widehat{D}} \simi_d(C^\prime, D^\prime)\right)}%
  %     {\sum_{C^\prime \in \widehat{C}}g(C^\prime)} \quad \text{if}\; C \ne \top \label{simi:general}\\
  %     \simi_d(A,B) &= pm(A,B) \quad \text{for atoms}\; A,B \\
  %     \simi_d(\exists{}r.C,\exists{}s.D) &=
  %     pm(r,s) \cdot \left(w + (1-w)\simi_d(C,D)\right) \\
  %     \simi_d(C,D) &= 0 \quad \text{otherwise}.
  %   \end{align}
  % \end{definition}
\end{frame}
\end{document}
