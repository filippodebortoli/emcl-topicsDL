% arara: xelatex
% arara: xelatex

%\documentclass[handout]{beamer}
\documentclass[smaller]{beamer}

\usepackage{microtype}
\usepackage{xcolor,xspace}
\usepackage{appendixnumberbeamer}
\usepackage{notation}
\usepackage{textcomp}
\usepackage{tcolorbox}
\tcbuselibrary{theorems}

\usetheme[numbering=fraction,%
       block=fill,%
       progressbar=foot%
       ]{metropolis} % Use metropolis theme
\setbeamercovered{transparent}

\newcounter{simprop}
\newcommand{\propstop}{\setcounter{simprop}{\theenumi}}
\newcommand{\propplay}{\setcounter{enumi}{\thesimprop}}

\title{A Framework for Semantic-based Similarity Measures for ELH-Concepts}
\author{Filippo De Bortoli}
\institute{European Master's Program in Computational Logic, TU Dresden}
\date{January 18th, 2018}

\begin{document}
\maketitle

\begin{frame}
  \frametitle{Covered paper}
  Assigned paper:
  \begin{thebibliography}
    \bibitem{}
     Lehmann, K. \& Turhan, A.
	   \newblock{\em A Framework for Semantic-Based Similarity Measures for ELH-Concepts}.
	   \newblock{\bf Logics in Artificial Intelligence - 13th European Conference}, {JELIA} , 2012.
  \end{thebibliography}
\end{frame}

\begin{frame}
  \frametitle{Concept similarity measures (\csm)}
  \begin{itemize}
    \item Functions that assess similarity between concept descriptions
    \item Several applications, including:
    \begin{itemize}
      \item Bioinformatics: Gene Ontology
      \note{Bioinformatics: Semantic similarity measures employed to compare genes and proteins wrt to their functions, rather than their sequence similarity.}
      \item Geoinformatics: \textsc{sim-dl}
      \note{Geoinformatics: Similarity measures used to improving information retrieval and integration of heterogeneous spatial data sources.}
      \item Ontology alignment
      \note{Ontology alignment: similarity measures used to determine corrispondences between concepts in different ontologies.}
    \end{itemize}
    \item Based on concept structure or concept semantics
  \end{itemize}
\end{frame}

\begin{frame}
  \frametitle{Motivation for a new concept similarity measure}
  
  \begin{columns}
    \begin{column}{0.55\textwidth}
      \begin{enumerate}[<+->]
        \item Are \(Mother\) and
        \(Woman \sqcap \exists{}parentOf.Person\) are similar?
        \item Are \(Mother\) and \(Father\) as similar
        as \(Woman \sqcap Parent\) and \(Father\) are?
      \end{enumerate}
    \end{column}
    \begin{column}{0.45\textwidth}
        \begin{align*}
          Woman &\subsume{} Person \\
          Man &\subsume{} Person \\
          Parent &\equiv \exists{}parentOf.Person \\
          Mother &\equiv Woman \sqcap Parent \\
          Father &\equiv Man \sqcap Parent \\
          &\vdots
        \end{align*}
    \end{column}
  \end{columns}
  \begin{itemize}[<+->]
    \item Equivalence \alert{invariance} and \alert{closure}
    of concept similarity measures (\csm)
    recognised as important properties\ldots
    \item \ldots yet, many existing
    \csm{}s fail to accomplish them!
  \end{itemize}
\end{frame}

\begin{frame}
  \frametitle{Overview}
  \begin{enumerate}
    \item Preliminaries
    \begin{enumerate}
      \item The \elh Description Logic
      \item \elh-normal form
    \end{enumerate}
    \item Similarity in Description Logic
    \begin{enumerate}
      \item Properties of similarity measures
    \end{enumerate}
    \item The concept similarity measure \(\simi\)
    \begin{enumerate}
      \item Derivation of \(\simi\)
      \item Definition of \(\simi\)
      \item Properties of \(\simi\)
    \end{enumerate}
    \item Conclusion
  \end{enumerate}
\end{frame}

\section{Preliminaries}

\begin{frame}
  \frametitle{The \elh{} Description Logic}
  \begin{columns}
    \begin{column}{0.5\textwidth}
      \el concept descriptions
      \begin{itemize}[<+->]
        \item \(\Conc\) \alert{concept names}, \(\R\) \alert{roles}
        \item \(C ::= \top \mid A \mid C \sqcap D \mid \exists{}r.C\)
        \item \alert{TBox} a finite set of axioms \(A \subsume{} C\), \(C \equiv D\)
        \item \(\I \models C \subsume{} D\) iff \(C^\I \subseteq D^\I\)
      \end{itemize}
    \end{column}
    \begin{column}{0.5\textwidth}
      \elh = \el + \alert{role hierarchy}
      \begin{itemize}[<+->]
        \item \alert{Role inclusion axioms} (\ria{}s) \(r \sqsubseteq s\)
        \item \alert{RBox} as a finite set of \ria{}s
        \item Given an interpretation \(\I\), \(\I \models r \sqsubseteq s\) iff \(r^\I \subseteq s^\I\)
      \end{itemize}
    \end{column}
  \end{columns}
  \pause
  \textbf{Knowledge base} \(\mathcal{K} = (\mathcal{T}, \mathcal{R})\), with \(\mathcal{T}\) a TBox, \(\mathcal{R}\) an RBox.
  
  \pause
  \textbf{Assumption:} The TBox is \alert{unfoldable}.
  \begin{itemize}[<+->]
    \item Acyclic
    \item Only contains axioms of the form
    \(A \sqsubseteq C\) or \(A \equiv C\), with \(A \in \Conc\)
    \item Each concept name at most once in LHS of an axiom.
    \item This allows to operate without the TBox!
  \end{itemize}
\end{frame}

\begin{frame}
  \frametitle{Concepts, atoms and normal form}
  For a \elh-concept in conjunction form
  \(C = \bigsqcap_{i \le n} C_i\), define
  \(
    \atom{C} := \lbrace C_1,\dotsc,C_n \rbrace
  \)
  as the set of its \alert{atoms}.
  \note{Intuitively, atoms are concepts not in
  conjunction form.}

  \elh-concepts admit a unique (mod. commutativity) \alert{normal form}!

  This is obtained by applying exhaustively:
  \begin{itemize}[<+->]
    \item \(A \sqcap \top \mapsto A\)
    \item \(A \sqcap A \mapsto A\)
    \item \(\exists{}r.C^\prime \mapsto%
           \exists{}ch([r]).C^\prime\),
           where \(ch\) is a role choice function
    \item \(\exists{}r.C^\prime \sqcap%
           \exists{}s.D^\prime \mapsto%
          \exists{}r.C^\prime\) if
          \(r\subsume{K}s\) and
          \(C^\prime \subsume{K} D^\prime\)
  \end{itemize}
\end{frame}


\section{Similarity measures}

\begin{frame}
  \frametitle{Metric spaces and similarity}
  \begin{enumerate}
    \item Metric spaces arise in many practical
    applications!
    \item Metrics as \alert{dissimilarity} measures!
    \item Given a metric \(d\) over a normalized space,
    the corresponding \alert{similarity} measure is
    \(
      s(a,b) := 1 - d(a,b)
    \).
  \end{enumerate}  
\end{frame}

\begin{frame}
  \frametitle{Properties of \csm{}s}
  Desired properties for a 
  \alert{similarity measure} \(\sme\) (where \(C,D,E\) are \elh-concepts):
  \begin{enumerate}[<+->]
    \item \alert{Symmetry}: \(\sme(C,D) = \sme(D,C)\)
    \item \alert{Triangle inequality}:
      \(1 + \sme(D,E) \ge \sme(D,C) + \sme(C,E)\)
    \item\label{equ:1} \alert{Equivalence closed}:
      \(\sme(C,D) = 1\) iff \(C \equiv D\)
    \item\label{equ:2} \alert{Equivalence invariant}:
      if \(C \equiv D\), then
      \(\sme(C,E) = \sme(D,E)\)
    \propstop
  \end{enumerate}
  Properties~\ref{equ:1} and~\ref{equ:2} are the most desirable one!
  \begin{itemize}
    \item Equivalent concepts should be similar (property~\ref{equ:1})
    \item They should similarly relate to other concepts (property~\ref{equ:2})
  \end{itemize}
\end{frame}

\begin{frame}
  \frametitle{Properties of similarity, ctd.}
  New properties are introduced:
  \begin{enumerate}[<+->]
    \propplay
    \item \alert{\(\subsume{}\)-preserving}: if \(C \subsume{} D \subsume{} E\),
    then \(\sme(C,D) \ge \sme(C,E)\)
    \item \alert{Reverse \(\sqsubseteq\)-preserving}: 
    \(\sme(C,E) \le \sme(D,E)\) if
    \(C \subsume{} D \subsume{} E\)
    \item \alert{Structurally dependent}: 
    Let \({(C_n)}_n \subseteq \A\) of mutually
    non-subsuming atoms. Then, the concepts
    \(D_n\), \(E_n\) satisfy
    \(\lim_{n \to \infty} \sme(D_n,E_n) = 1\)
    \begin{equation*}
      D_n := \bigsqcap_{i \le n} C_i \sqcap D \qquad
      E_n := \bigsqcap_{i \le n} C_i \sqcap E
    \end{equation*}
    Intuitively, the more \alert{features}
    two concepts share, the more similar they are.
  \end{enumerate}
\end{frame}

\section{Deriving simi}

\begin{frame}
  \frametitle{Starting point for the derivation of \(\simi\): a set measure}
  First, introduce a variation of the \alert{Jaccard Index}:
  \begin{equation}
    d(C,D) := \frac{\abs{\atom{C} \cap \atom{D}}}%
                   {\abs{\atom{C}}}
  \end{equation}
  What about the numerator?
  \begin{equation}
    \abs{\atom{C} \cap \atom{D}} =
    \sum_{C^\prime \in \atom{C}}%
      \textcolor{cyan}{\max_{D^\prime \in \atom{D}}}%
      (\mydelta_{C^\prime}^{D^\prime})
    \qquad \mydelta_{C^\prime}^{D^\prime}
    = \begin{cases}
      1 & C^\prime = D^\prime \\ 0 & C^\prime \ne D^\prime
    \end{cases}
  \end{equation}
  \pause
  The function \(\mydelta\) can be improved:
  \begin{enumerate}[<+->]
    \item Treats different concept names as totally dissimilar
    \item Does not take into account role names
    \item Cannot properly treat existential restrictions
  \end{enumerate}
\end{frame}

\begin{frame}
  \frametitle{Generalizing \(\delta\) to a primitive measure for names}
  \begin{definition}[Primitive measure]
    Mapping \(\mypm \colon {(\Conc \cup \R)}^2 \to [0,1]\)
    satisfying:
    \begin{enumerate}[<+->]
      \item\label{pm:1} \(\mypm(A,B) = 1\) iff \(A = B\)
      \item\label{pm:2} \(\mypm(r,s) = 1\) iff \(s \subsume{K} r\)
      \item\label{pm:3} if \(r \subsume{K} s\), then \(\mypm(r,s) > 0\),
      \item\label{pm:4} if \(t \subsume{K} s\), then \(\mypm(r,s) \le \mypm(r,t)\).
    \end{enumerate}
    for \(A,B \in \Conc\) and \(r,s,t \in \R\).
  \end{definition}
  \begin{itemize}[<+->]
    \item \ref{pm:1} +~\ref{pm:2} \textrightarrow Eq. closure
    \item Rule~\ref{pm:4} guarantees
    \(\subsume{}\)-preserving
  \end{itemize}
\end{frame}

\begin{frame}
  \frametitle{Handling existential restrictions}
  Case analysis (here, \(A,B \in \Conc\),
  \(C,D \in \celh\),
  \(r \in \R\)):
  \begin{enumerate}[<+->]
    \item \(d(A,B) = \mypm(A,B)\)
    \item \(d(\exists{}r.C,D) = d(D,\exists{}r.C) = 0\)
    \item \(d(\exists{}r.C,\exists{}s.D) =
    \mypm(r,s) \cdot \left(\myw + (1-\myw) \cdot d(C,D)\right)\)
  \end{enumerate}

  Parameter \(\myw \in (0,1)\) allows
  to choose whether roles or concepts account more
  in the similarity measure.
\end{frame}

\begin{frame}
  \frametitle{Does ``max'' perform well w.r.t. similarity?}
  The \(\textcolor{cyan}{\max}\) operator is the weakest \alert{t-conorm}.
  \begin{itemize}
    \item It is possible to make better use of
          the given information!
  \end{itemize}
  \pause
  \begin{definition}[Bounded t-conorm]
    Operator \(\myconorm \colon {[0,1]}^2 \to [0,1]\) which is
    commutative, associative and for all
    \(x, y, z, w \in [0,1]\):
    \begin{enumerate}[<+->]
      \item \alert{Unit element}: \(0 \myconorm x = x = x \myconorm 0\)
      \item \alert{Monotonicity}: if \(x \le z\) and \(y \le w\), then \(x \myconorm y \le z \myconorm w\)
      \item \alert{Boundedness}: if \(x \myconorm y = 1\), then \(x = 1\) or \(y = 1\).
    \end{enumerate}
  \end{definition}
\end{frame}

\begin{frame}
  \frametitle{Which concepts are more relevant?}
  Some atoms \textit{may be} more important than others
  \begin{itemize}
    \item Contextualization
    \item Priority assignment
  \end{itemize}
  \begin{definition}[Weighting function]
    A \emph{weighting function} is a mapping 
    \(\myg \colon \mathbf{A} \to \mathbb{R}_+\),
     where \(\mathbf{A}\) is the set of \emph{atoms}.
  \end{definition}
  With this definition, it is possible to partially
  ensure structural dependence for the similarity
  measure.
\end{frame}

\section{Definition of simi}

\begin{frame}
  \frametitle{Directed similarity}
  \alert{Directed similarity} as a measure of subsumption:
  if \(\sme_d\) measures directed similarity and \(\sme_d(C,D) = 1\), then \(D \subsume{} C\).

  Introducing a measure of directed similarity:
  \begin{equation}
  \tcboxmath{\simi_d \colon \celh^2 \to [0,1]}
  \end{equation}
  How does \(\simi_d\) behave?
  \begin{equation}
    \simi_d(\top,D) = 1 \qquad D \in \celh
  \end{equation}
  Thus, \(D \sqsubseteq \top\) for all \elh-concepts.
  \begin{equation}
    \simi_d(A,B) = \mypm(A,B) \qquad A,B \in \Conc
  \end{equation}
  Concept names are measured according to \(\mypm\).
\end{frame}

\begin{frame}
  \frametitle{Directed similarity, ctd.}
  \begin{equation}
    \simi_d(\exists{}r.C,\exists{}s.D) =
    \mypm(r,s) \cdot 
    \left(\myw + 
    (1-\myw)\simi_d(C,D)\right)
  \end{equation}

  The general case, for \(C,D \ne \top\), is
  \begin{equation}
    \simi_d(C,D) =
    \frac{\sum_{C^\prime \in \atom{C}}\left(\myg(C^\prime) \cdot \textcolor{cyan}{\bigoplus_{D^\prime \in \atom{D}}} \simi_d(C^\prime, D^\prime)\right)}%
    {\sum_{C^\prime \in \atom{C}}\myg(C^\prime)}
  \end{equation}
  Otherwise, \(\simi_d(C,D) = 0\).
\end{frame}

\begin{frame}
  \frametitle{Fuzzy connectors and simi}
  \begin{equation}
    \tcboxmath{\simi(C,D) := \simi_d(C,D) \otimes \simi_d(D,C)}
  \end{equation}
  Recall that \(C \equiv D\) iff \(C \subsume{} D\)
  and \(D \subsume{} C\).
  \begin{definition}[Fuzzy connector]
    Commutative operator
    \(\otimes \colon%
    {[0,1]}^2 \to [0,1]\) s.t. for all
    \(x, y \in [0,1]\),
    \begin{enumerate}[<+->]
      \item\label{fu:1} \alert{Equivalence closed}:
      \(x \otimes y = 1\) iff \(x = y = 1\)
      \item\label{fu:2} \alert{Weak monotonic}:
      if \(x \le y\), then
      \(1 \otimes x \le 1 \otimes y\) 
      \item\label{fu:3} \alert{Bounded}:
      if \(x \otimes y = 0\), then
      \(x = 0\) or \(y = 0\)
      \item\label{fu:4} \alert{Grounded}:
      \(0 \otimes 0 = 0\)
    \end{enumerate}
  \end{definition}
\end{frame}

\begin{frame}
  \frametitle{Properties of simi}
  \begin{theorem}[Properties of \(\simi\)]
    The function \(\simi\) is
    symmetric,
    equivalence invariant,
    equivalence closed and
    \(\subsume{}\)-preserving
    for all choices of its parameters.
  \end{theorem}
  \begin{theorem}[Computation of \(\simi\)]
    The function \(\simi\) can be computed in time
    polynomial to the size of the measured concepts,
    if all its parameters can be computed in polynomial time.
  \end{theorem}
  \begin{itemize}
    \item The similarity measure \(\simi\)
    achieved a large number of desired properties
    (but not all of them\ldots)
    \item Moreover, it can be computed efficiently!
  \end{itemize}
\end{frame}

\section{Conclusion}

\begin{frame}
  \frametitle{How to go on from here?}
  \begin{enumerate}[<+->]
    \item Can the \(\simi\) approach be generalized
    (e.g. general TBoxes)?
    \item Can it be extended to
    more expressive \dl{}s?
    \item Can one guarantee more properties
    (e.g. triangle inequality) without losing
    equivalence-related ones?
    \item How can \(\simi\) be implemented?
  \end{enumerate}
\end{frame}

\begin{frame}[standout]
  Thank you!
\end{frame}
\end{document}
