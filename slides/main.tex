%\documentclass[handout]{beamer}
\documentclass{beamer}

\usepackage{microtype}
\usepackage{xcolor,xspace}
\usepackage{appendixnumberbeamer}
\usepackage{notation}

%\usetheme[numbering=fraction,%
%       block=fill,%
%       progressbar=foot%
%       ]{metropolis} % Use metropolis theme
\useoutertheme[numbering=fraction]{metropolis}
\useinnertheme{metropolis}
\usefonttheme{metropolis}
\usecolortheme{metropolis}
%\usecolortheme{spruce}
\setbeamercovered{transparent}
\title{A Framework for Semantic-based Similarity Measures for ELH-Concepts}
\author{Filippo De Bortoli}
\institute{European Master's Program in Computational Logic, TU Dresden}
\date{January 18th, 2018}

\begin{document}
\maketitle

\begin{frame}
  \frametitle{Something about the original work\ldots}
  This presentation originates from:
  \begin{thebibliography}
    \bibitem{}
     Lehmann, K. \& Turhan, A.
	   \newblock{\em A Framework for Semantic-Based Similarity Measures for ELH-Concepts}.
	   \newblock{\bf Logics in Artificial Intelligence - 13th European Conference}, {JELIA} , 2012.
  \end{thebibliography}
\end{frame}

\begin{frame}
  \frametitle{Motivation for a new concept similarity measure}
  
  \begin{columns}
    \begin{column}{0.61\textwidth}
      \begin{enumerate}[<+->]
        \item Would you claim that \(Mother\) and
        \(Woman \sqcap \exists{}parentOf.Person\) are similar?
        \item Are \(Mother\) and \(Father\) as similar
        as \(Woman \sqcap Parent\) and \(Father\) are?
      \end{enumerate}
      \begin{itemize}[<+->]
        \item Equivalence \alert{invariance} and \alert{closure}
        of concept similarity measures (\csm)
        recognised as important properties\ldots
        \item \ldots yet, many existing
        \csm{}s fail to accomplish them!
        %\footnote{Despite claiming those properties to hold.}
      \end{itemize}
    \end{column}
    \begin{column}{0.39\textwidth}
      \begin{align*}
        Parent &\equiv \exists{}parentOf.Person \\
        Mother &\equiv Woman \sqcap Parent \\
        Woman &\sqsubseteq Person \\
        &\vdots         
      \end{align*}
    \end{column}
  \end{columns}
\end{frame}

\begin{frame}
  \frametitle{Overview}
  \begin{enumerate}
    \item Preliminaries
    \begin{enumerate}
      \item The \elh Description Logic
      \item \elh-normal form
    \end{enumerate}
    \item Similarity in Description Logic
    \begin{enumerate}
      \item Properties of a similarity measure
    \end{enumerate}
    \item A new concept similarity measure
    \begin{enumerate}
      \item Derivation of \(\simi\)
      \item Definition of \(\simi\)
      \item Properties of \(\simi\)
    \end{enumerate}
    \item Conclusion
  \end{enumerate}
\end{frame}

\section{Preliminaries}

\begin{frame}
  \frametitle{The \elh{} Description Logic}
  \elh = \el + \alert{role hierarchy}
  \begin{itemize}[<+->]
    \item \alert{Role inclusion axioms} (\ria{}s) \(r \sqsubseteq s\)
    \item \alert{RBox} as a finite set of \ria{}s
    \item \(\I \models r \sqsubseteq s\) iff \(r^\I \subseteq s^\I\)
  \end{itemize}
  \textbf{Assumption:} The TBox is \alert{unfoldable}.
  \begin{itemize}[<+->]
    \item It is acyclic
    \item It only contains axioms of the form
    \(A \sqsubseteq C\) or \(A \equiv C\), with \(A \in \Conc\)
    \item Each concept name at most once in LHS of an axiom.
    \item This allows to operate without the TBox!
  \end{itemize}
\end{frame}

\begin{frame}
  \frametitle{Concepts, atoms and normal form}
  For a \elh-concept in conjunction form
  \(C = \bigsqcap_{i \le n} C_i\), define
  \(
    \atom{C} := \lbrace C_1,\dotsc,C_n \rbrace
  \)
  as the set of its \alert{atoms}.
  Intuitively, atoms are concepts not in
  conjunction form.

  \elh-concepts admit an essentially unique \alert{normal form}!

  This is obtained by applying exhaustively:
  \begin{itemize}[<+->]
    \item \(A \sqcap \top \mapsto A\)
    \item \(A \sqcap A \mapsto A\)
    \item \(\exists{}r.C^\prime \mapsto%
           \exists{}ch([r]).C^\prime\),
           where \(ch\) is a role choice function
    \item \(\exists{}r.C^\prime \sqcap%
           \exists{}s.D^\prime \mapsto%
          \exists{}r.C^\prime\) if
          \(r\subsume{K}s\) and
          \(C^\prime \subsume{K} D^\prime\)
  \end{itemize}
\end{frame}


\section{What is similarity?}

\begin{frame}
  \frametitle{Metric spaces and similarity}
  \begin{enumerate}
    \item Metric spaces arise in many practical
    applications!
    \item Metrics as \alert{dissimilarity} measures!
    \item Given a metric \(d\) over a normalized space,
    the corresponding \alert{similarity} measure is
    \(
      s(a,b) := 1 - d(a,b)
    \).
  \end{enumerate}  
\end{frame}

\begin{frame}
  \frametitle{Properties of similarity}
  What are possible properties for a
  \alert{similarity measure} \(\sme\)?

  In the following, \(C,D,E\) are \elh-concepts.

  Which properties were already in the literature?

  \begin{description}[<+->]
    \item[Symmetric] \(\sme(C,D) = \sme(D,C)\)
    \item[Triangle inequality]
      \(1 + \sme(D,E) \ge \sme(D,C) + \sme(C,E)\)
    \item[Equivalence invariant]
      if \(C \equiv D\), then
      \(\sme(C,E) = \sme(D,E)\)
    \item[Equivalence closed]
      \(\sme(C,D) = 1\) iff \(C \equiv D\)
  \end{description}
  The last two properties are the one that should
  be guaranteed!
\end{frame}

\begin{frame}
  \frametitle{Properties of similarity, ctd.}

  Which properties had been newly introduced?
  \begin{description}[<+->]
    \item[\(\sqsubseteq\)-preserving] if \(C \sqsubseteq D \sqsubseteq E\),
    then \(\sme(C,D) \ge \sme(C,E)\)
    \item[Reverse \(\sqsubseteq\)-preserving]
    \(\sme(C,E) \le \sme(D,E)\) if
    \(C \sqsubseteq D \sqsubseteq E\)
    \item[Structurally dependent]
    Let \({(C_n)}_n \subseteq \A\) of mutually
    non-subsuming atoms. Then, the concepts
    \(D_n\), \(E_n\) satisfy
    \(\lim_{n \to \infty} \sme(D_n,E_n) = 1\)
    \begin{equation*}
      D_n := \bigsqcap_{i \le n} C_i \sqcap D \qquad
      E_n := \bigsqcap_{i \le n} C_i \sqcap E
    \end{equation*}
  \end{description}
  \begin{itemize}[<+->]
    \item Intuitively, the more \alert{features}
    two concepts share, the more similar they are.
  \end{itemize}
\end{frame}

\section{Deriving simi}

\begin{frame}
  \frametitle{Starting point for the derivation of \(\simi\)}
  First, introduce a variation of the \alert{Jaccard Index}:
  \begin{equation}
    d(C,D) := \frac{\abs{\atom{C} \cap \atom{D}}}%
                   {\abs{\atom{C}}}
  \end{equation}
  What about the numerator?
  \begin{equation}
    \abs{\atom{C} \cap \atom{D}} =
    \sum_{C^\prime \in \atom{C}}%
      \textcolor{cyan}{\max_{D^\prime \in \atom{D}}}%
      (\textcolor{red}{\delta}_{C^\prime}^{D^\prime})
  \end{equation}
  Here, \(\textcolor{red}{\delta}_{C^\prime}^{D^\prime} = 1\) if
  \(C^\prime = D^\prime\), \(0\) otherwise.
  \pause
  \begin{enumerate}[<+->]
    \item Are different concept names totally dissimilar?
    \item Does \(\textcolor{red}{\delta}\) say something about roles?
    \item Does it take into account existential restrictions?
  \end{enumerate}
\end{frame}

\begin{frame}
  \frametitle{Generalizing \(\delta{}\) to a primitive measure for names}
  \begin{definition}[Primitive measure]
    Mapping \(\textcolor{red}{pm} \colon {(\Conc \cup \R)}^2 \to [0,1]\)
    satisfying:
    \begin{enumerate}[<+->]
      \item\label{pm:1} \(\textcolor{red}{pm}(A,B) = 1\) iff \(A = B\)
      \item\label{pm:2} \(\textcolor{red}{pm}(r,s) = 1\) iff \(s \sqsubseteq r\)
      \item\label{pm:3} if \(r \subsume{R} s\), then \(\textcolor{red}{pm}(r,s) > 0\),
      \item\label{pm:4} if \(t \subsume{R} s\), then \(\textcolor{red}{pm}(r,s) \le \textcolor{red}{pm}(r,t)\).
    \end{enumerate}
    for \(A,B \in \Conc\) and \(r,s,t \in \R\).
  \end{definition}
  \begin{itemize}[<+->]
    \item Rules~\ref{pm:1} and~\ref{pm:2} will 
    ensure equivalence closure
    \item Rule~\ref{pm:4} guarantees
    \(\subsume{}\)-preserving
  \end{itemize}
\end{frame}

\begin{frame}
  \frametitle{How should one handle existential restrictions?}
  It is possible to make a case analysis about
  concepts which similarity one wants to compute.
  \begin{enumerate}[<+->]
    \item \(d(A,B) = \textcolor{red}{pm}(A,B)\)
    \item \(d(\exists{}r.C,D) = d(D,\exists{}r.C) = 0\)
    \item \(d(\exists{}r.C,\exists{}s.D) =
    \textcolor{red}{pm}(r,s) \cdot \left(w + (1-w) \cdot d(C,D)\right)\)
  \end{enumerate}
  for \(A,B \in \Conc\),
  \(C,D \in \mathcal{C}(\mathcal{ELH})\),
  \(r \in \R\).

  Here, \(\textcolor{magenta}{w} \in (0,1)\) is a parameter which allows
  to choose whether roles or concepts account more
  in the similarity measure.
\end{frame}

\begin{frame}
  \frametitle{Does ``max'' perform well w.r.t. similarity?}
  The \(\textcolor{cyan}{\max}\) operator is the weakest \alert{t-conorm}.
  \begin{itemize}
    \item It is possible to make better use of
          the given information!
  \end{itemize}
  \pause
  \begin{definition}[Bounded t-conorm]
    An operator \(\oplus \colon {[0,1]}^2 \to [0,1]\) which is
    commutative, associative and for all
    \(x, y, z, w \in [0,1]\):
    \begin{description}[<+->]
      \item[Unit element] \(0 \oplus x = x = x \oplus 0\)
      \item[Monotonicity] if \(x \le z\) and \(y \le w\), then \(x \oplus y \le z \oplus w\)
      \item[Boundedness] if \(x \oplus y = 1\), then \(x = 1\) or \(y = 1\).
    \end{description}
  \end{definition}
\end{frame}

\begin{frame}
  \frametitle{Which concepts are more relevant?}
  One may want to impose that some atoms are
  more important than others, in measuring similarity!
  \begin{definition}[Weighting function]
    A \emph{weighting function} is a mapping 
    \(\textcolor{blue}{g} \colon \mathbf{A} \to \mathbb{R}_+\),
     where \(\mathbf{A}\) is the set of \emph{atoms}.
  \end{definition}
  With this definition, it is possible to partially
  ensure structural dependence for the similarity
  measure.
\end{frame}

\section{Definition of simi}

\begin{frame}
  \frametitle{Directed similarity}
  Finally, a measure of \alert{directed similarity} is defined.
  \begin{equation}
    \simi_d \colon {\mathcal{C}(\mathcal{ELH})}^2%
    \to [0,1]
  \end{equation}
  How does \(\simi_d\) behave?
  \begin{equation}
    \simi_d(\top,D) = 1 \qquad D \in \mathcal{C}(\mathcal{ELH})
  \end{equation}
  This captures the following fact: \(D \sqsubseteq \top\)
  for all \elh-concepts.
  \begin{equation}
    \simi_d(A,B) = \textcolor{red}{pm}(A,B) \qquad A,B \in \Conc
  \end{equation}
  Concept names are measured according to \(\textcolor{red}{pm}\).
\end{frame}

\begin{frame}
  \frametitle{Directed similarity, ctd.}
  \begin{equation}
    \simi_d(\exists{}r.C,\exists{}s.D) =
    \textcolor{red}{pm}(r,s) \cdot 
    \left(\textcolor{magenta}{w} + 
    (1-\textcolor{magenta}{w})\simi_d(C,D)\right)
  \end{equation}
  Existential restrictions are measured in an
  appropriate way.

  The general case, for \(C,D \ne \top\), is
  \begin{equation}
    \simi_d(C,D) =
    \frac{\sum_{C^\prime \in \widehat{C}}\left(\textcolor{blue}{g}(C^\prime) \cdot \textcolor{cyan}{\bigoplus_{D^\prime \in \widehat{D}}} \simi_d(C^\prime, D^\prime)\right)}%
    {\sum_{C^\prime \in \widehat{C}}\textcolor{blue}{g}(C^\prime)}
  \end{equation}
  Otherwise, \(\simi_d(C,D) = 0\).
\end{frame}

\begin{frame}
  \frametitle{Fuzzy connectors and simi}
  Recall that \(C \equiv D\) iff \(C \subsume{} D\)
  and \(D \subsume{} C\).
  Mathematically, this is reflected on...
  \begin{definition}[Fuzzy connector]
    Commutative operator
    \(\otimes \colon%
    {[0,1]}^2 \to [0,1]\) s.t. for all
    \(x, y \in [0,1]\),
    \begin{description}[<+->]
      \item[Equivalence closed]\label{fu:1}
      \(x \otimes y = 1\) iff \(x = y = 1\)
      \item[Weak monotonic]\label{fu:2}
      if \(x \le y\), then
      \(1 \otimes x \le 1 \otimes y\) 
      \item[Bounded]\label{fu:3}
      if \(x \otimes y = 0\), then
      \(x = 0\) or \(y = 0\)
      \item[Grounded]\label{fu:4}
      \(0 \otimes 0 = 0\)
    \end{description}
  \end{definition}
  Then, \(\simi\) is defined as
\(\simi(C,D) := \simi_d(C,D) \otimes \simi_d(D,C)\).
\end{frame}

\begin{frame}
  \frametitle{Properties of simi}
  \begin{theorem}[Properties of \(\simi\)]
    The function \(\simi\) is
    symmetric,
    equivalence invariant,
    equivalence closed and
    \(\subsume{}\)-preserving
    for all choices of its parameters.
  \end{theorem}
  \begin{theorem}[Computation of \(\simi\)]
    The function \(\simi\) can be computed in time
    polynomial to the size of the measured concepts,
    if all its parameters can be computed in polynomial time.
  \end{theorem}
  \begin{itemize}
    \item The similarity measure \(\simi\)
    achieved a large number of desired properties
    (but not all of them\ldots)
    \item Moreover, it can be computed efficiently!
  \end{itemize}
\end{frame}

\section{Conclusion}

\begin{frame}
  \frametitle{How to go on from here?}
  \begin{enumerate}[<+->]
    \item Can the \(\simi\) approach be generalized
    (e.g. general TBoxes)?
    \item Can it be extended to
    more expressive \dl{}s?
    \item Can one guarantee more properties
    (e.g. triangle inequality) without losing
    equivalence-related ones?
    \item How can \(\simi\) be implemented?
  \end{enumerate}
\end{frame}

\begin{frame}[standout]
  Thank you!
\end{frame}
%
%\begin{frame}
%  \frametitle{Similarity measures and the real world}
%  \begin{enumerate}
%    \item \emph{KB exploration}, e.g. discovery of genetic functionality;
%    \item \emph{Ontology alignment}, i.e. how to match different ontologies.
%  \end{enumerate}
%\end{frame}
%
\end{document}
