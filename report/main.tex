\let\oldvec\vec%
\documentclass{llncs}
\let\vec\oldvec

\usepackage[T1]{fontenc}
\usepackage[utf8]{inputenc}
\usepackage[english]{babel}
\usepackage{microtype}

%\usepackage[backend=biber,style=lncs]{biblatex}
%  \addbibresource{seminar_further-DL.bib}
\bibliographystyle{splncs03}
\usepackage{csquotes}
\usepackage{xspace}

\usepackage{amsmath}
\usepackage{amssymb}


\usepackage{booktabs}
\setlength{\tabcolsep}{7pt}
%\usepackage{multirow}

\usepackage{todonotes}

\usepackage{hyperref}

\usepackage{notation}

\title{A Framework for Semantic-based Similarity Measures for \elh-Concepts}
\author{Filippo De Bortoli}
\institute{European Master's Program in Computational Logic, TU Dresden}% \and \email{filippo.de_bortoli@mailbox.tu-dresden.de}}

\begin{document}
  \maketitle

  \begin{abstract}
    Measuring similarity of concept descriptions in different Description Logics is a challenging task, especially in expressive ones.
    Moreover, many of the measures devised in the past years had been often defined with respect of a particular cluster of instance data, or they only considered the syntactical aspects of the concept descriptions.

    In this report, the \(\simi\) framework for designing concept similarity measures will be introduced and discussed, from desired properties of such functions to actual design choices.
    Furthermore, a brief perspective into related work will be given, with concern to unsolved problems of \(\simi\) and general challenges for concept similarity measures.
  \end{abstract}

  \section{Introduction}
\label{sec:intro}

Measuring the similarity of two or more concept descriptions is a task that frequently occurs in several real-world cases.
These applications, where ontologies are employed, range from exploration of knowledge bases for information discovery --- as done in~\cite{GF13} --- to more technical activies, like ontology alignment, as in~\cite{CHu11}.


As important as a \emph{good} measure for concept similarity is, one has to deal with several issues during the specification, implementation and usage phases of a tool of this kind.
Most of these have been faced in the work done by Lehmann \& Turhan in~\cite{LeTu12}.

Which are the intended properties for a concept similarity measure?
The notion of similarity itself has been largely discussed in the literature (a remarkable reference is the work done in~\cite{Tve77}).
Among the many, possibly different, existing approaches to similarity, the one shows in [] has been followed in this work.
In particular, relevant properties of similarity have been described and formalised; some of these have been drawn from what has been already established in the literature, whereas others have been newly introduced.
These properties are either concerning the mathematical side of a measure or its relation with the underlying representation of the concepts.

Description Logics (\dl) are a family of formal knowledge representation languages.
The \dl \elh, upon which \(\simi\) has been defined, has a practical relevance: for example, the medical ontology \textsc{snomed ct}~\cite{snomed} employs a variant of \elh, and the Web standard \textsc{Owl-el}, described in~\cite{owlEL}, is originated from \elh.
In this setting, it is shown how a concept similarity measure can be defined, which achieves \emph{power} --- thanks to \emph{equivalence invariance} and \emph{closure} ---, \emph{flexibility} in the choice of its parameters and \emph{efficiency} in its computation, assuming that the right tools are used in its construction.

This work also addresses another issue: the \emph{portability} and \emph{reliability} of concept similarity measures, across different contexts.
Indeed, many existing measures are domain-related or depend too much on the dataset upon which they are first defined, assessed and evaluated.
The choice, here, is that of defining a framework, a general concept similarity measure which can be specifically instantiated, by means of several parameters; this allows the user to tailor the measure to her particular application, without losing the desired properties, guaranteed to hold by the way the introduce measure \(\simi\) has been specified.

The report has the following structure:
first, some notions concerning \dl{}s are introduced;
then, similarity measure are discussed, with their properties;
next, the framework devised in~\cite{LeTu12} is derived and explained;
finally, some results are shown, reguarding the properties of the new measure.
In the end, an overview of related works is presented, followed by a brief conclusion.

  \section{The \(\mathcal{ELH}\) Description Logic}

Hereafter, the reader must assume that the \elh description logic (\dl) is used where not differently specified.
For a detailed treatment of the notions here introduced, the interested reader can go to~\cite{DLbook}.

The syntax of a \dl makes use of two finite and disjoint sets of \emph{concept names}, \(\Conc\) and \emph{role names}, \(\R\) to build \emph{concept descriptions}, by means of \emph{concept constructors}.
The semantics of such descriptions is defined with an interpretation \(\I = (\DeltaI,\dotI)\), where \(\DeltaI \ne \emptyset\) is the \emph{interpretation domain} and \(\dotI\) is a mapping which assigns subsets of \(\DeltaI{}\) to each concept and binary relations over \(\DeltaI\) to roles.
The syntax and the semantics of the elements of the set of \emph{\elh-concept descriptions} \(\celh\) are introduced in the Table~\ref{tbl:el}.

\begin{table}
  \caption{Constructors of the \dl \el, along with their syntax and semantics w.r.t. an interpretation \(\I = (\DeltaI,\dotI)\). Here, \(A \in \Conc\), \(r \in \R\) and \(C, D \in \celh\).}
  \label{tbl:el}
  \centering
  \begin{tabular}{lcc}
    \toprule
    Name & Syntax & Semantics \\
    \midrule
    Top & \(\top\) & \(\DeltaI\) \\
    Concept name & \(A\) & \(A^\I\) \\
    Conjunction & \(C \sqcap D\) & \(C^\I \cap D^\I\)\\
    Existential restriction & \(\exists{}r.C\) &
    \(\lbrace d \in \DeltaI \mid \exists e \in \DeltaI.
    (d,e) \in r^\I \land e \in C^\I \rbrace\) \\
    \bottomrule
  \end{tabular}
\end{table}

In \elh, each concept description can be expressed in conjunctive form.
This feature will be later exploited to compute the value of \(\simi\) by means of the so-called \emph{atoms} of the concepts there compared.
\begin{definition}[Atoms]
  If \(\Conc\) and \(\R\) are the set of concept and role names (resp.), the set \(\A\) of \emph{\elh-atoms} (or \emph{atoms}) is defined as
  \begin{equation}
    \A := \Conc \cup \lbrace \top \rbrace \cup \lbrace \exists{}r.D \mid r \in \R, D \in \mathcal{C}(\mathcal{ELH}) \rbrace.
  \end{equation}
  Moreover, \(\widehat{\cdot} \colon \celh \to 2^{\celh}\) is defined as \(\widehat{C} := \lbrace C_i\rbrace_{i=1}^n\), where \(C = \bigsqcap_{i=1}^n C_i \in \celh\).
\end{definition}

The knowledge base (\kb) adopted hereafter has two components: the \emph{terminological} knowledge and another components, expressing information about the \emph{role hierarchy}.
These are represented by a \emph{TBox} \(\mathcal{T}\) and an \emph{RBox} \(\mathcal{R}\), finite sets of axioms, which syntax and semantics are defined in the Table~\ref{tbl:boxes}.
A concept name (resp. role name) that does not appear on the left-hand side (\lhs) of any axiom in a TBox (resp. RBox) is called \emph{primitive}.

\begin{table}
  \caption{Types of axioms that are found in a TBox or in an RBox, with their syntax and semantics w.r.t. an interpretation \(\I\). Here, \(A \in \Conc\), \(r,s \in \R\) and \(C, D \in \celh\).}
  \label{tbl:boxes}
  \centering
  \begin{tabular}{llcc}
    \toprule
    Family & Name & Syntax & Semantics \\
    \midrule
    TBox & General concept inclusion (\gci) & \(C \sqsubseteq D\) &\(C^\I \subseteq D^\I\) \\
    & Primitive concept definition & \(A \subsume{} C\) & \(A^\I \subseteq C^\I\) \\
    & Concept definition & \(A \equiv C\) & \(A^\I = C^\I\) \\
    \midrule
    RBox & Role inclusion axiom (\ria) & \(r \sqsubseteq s\) & \(r^\I \subseteq s^\I\) \\
    \bottomrule
  \end{tabular}
\end{table}

As an assumption that will make the definition of \(\simi\) easier, only a restricted class of TBoxes will be considered, namely, the \emph{unfoldable} TBoxes.
\begin{definition}[Unfoldable TBox]
  A TBox \(\mathcal{T}\) is called \emph{unfoldable} iff it is acyclic, consists of concept definitions and primitive concept definitions only and every concept name occurs at most once on a \lhs of a concept axiom.    
\end{definition}
An unfoldable TBox \(\mathcal{T}\) can be expanded in a TBox \(\mathcal{T}^\star\) which is semantical equivalent to it, contains (primitive) concept definitions, where only primitive concept names appear on the right-hand side of the concept definitions.
Additionally, primitive concept definitions can be transformed into concept definitions, adding new concept names to the TBox.
In the definition of \(\simi\), the input concepts will be expanded by replacing each defined concept with its definition in the TBox; consequently, only primitive concept names will appear and the knowledge contained in the TBox will not be needed.

An interpretation \(\I\) is a model for a TBox \(\mathcal{T}\) (resp. an RBox \(\mathcal{R}\)), in symbols \(\I \models \mathcal{T}\) (resp. \(\I \models \mathcal{R}\)), if it satisfies the semantic condition of all its \gci{}s and (primitive) concept definitions (resp. all its \ria{}s).

Among the various reasoning services that can be applied over \elh \kb{}s, a commonly used one is the \emph{subsumption} of concepts: a concept \(C\) is \emph{subsumed} by \(D\) w.r.t.\ a \kb \(\mathcal{K}\) iff \(C^\I \subseteq D^\I\) for all models \(\I\) of \(\mathcal{K}\), in symbols \(C \subsume{K} D\); if \(C\) and \(D\) both subsume each other, then they are \emph{equivalent}, in symbols \(C \equiv_{\mathcal{K}} D\).
Similarly, \(r \subsume{K} s\) if \(r^\I \subseteq s^\I\) for all models \(\I\) of \(\mathcal{R}\).

A normal form for \el concepts is well described in~\cite{DLbook}; in~\cite{LeTu12}, this is extended to take into account the content of the RBox.
Such a normal form can still be computed in polynomial time.
Specifically, a choice function \(ch{}\) is devised such that every role is replaced by a picked representative of its \(\equiv_\mathcal{R}\)-equivalence class.
\begin{definition}
  A concept \(C \in \mathcal{C}\)(\elh) is in \emph{\elh-normal form} w.r.t. a RBox \(\mathcal{R}\) if all its subconcepts are closed under the application of the following rules:
  \begin{align*}
    A \sqcap \top &\mapsto A & A \sqcap A &\mapsto A \tag{1--2}\\
    \exists{}r.C^\prime &\mapsto \exists{}ch([r]).C^\prime &
    \exists{}r.C^\prime \sqcap \exists{}s.D^\prime &\mapsto
    \exists{}r.C^\prime\,\,\text{if}\,\,r\subsume{R}s, C^\prime \sqsubseteq D^\prime \tag{3--4}
  \end{align*}
\end{definition}
  \section{Properties of similarity measures}
\label{sec:properties}
How may one quantitatively measure the amount of similarity between two concepts?
In many real-world application areas, ranging from cognitive sciences, biology, physics, to computer vision, the underlying mathematical structure is that of a \emph{metric space}, which is endowed with a \emph{metric}.
In particular, the similarity of two concepts can be interpreted as a relative measure: thus, it is possible to take normalised metric spaces, where distances are bound to live in the unit interval.
In this view, the metric distance between two concepts expresses their dissimilarity: thus, two totally similar concepts will be in the same position in the space.

\begin{definition}[(concept) similarity function]
  \label{dfn:sim}
  Given a normalised metric space \((D,d)\), the similarity function \(s\) induced by \(d\) corresponds to \(s(a,b) := 1 - d(a,b)\).

  Generally speaking, a \emph{similarity function} \(s \colon D^2 \to [0,1]\) over a normalizes space \(D\) satisfies these conditions, for all \(a,b,c \in D\):
  \begin{enumerate}
    \item \(s(a,b) = 1\) iff \(a = b\),
    \item \(s(a,b) = s(b,a)\),
    \item \(1 + s(a,b) \ge s(a,c) + s(c,b)\).
  \end{enumerate}
  A \emph{concept similarity measure} over \elh-concepts is defined as a symmetric mapping \(\sme \colon \celh^2 \to [0,1]\).
\end{definition}

  How much can be directly embedded in this definition?
  Which are the properties globally recognised as useful to measure concept similarity?
  A \emph{property}, here, is seen as a formalisation of the expected behaviour of the defined measure.
  Unsurprisingly, almost all the features, that one can think of as desirable for such a tool, have encountered some criticism in different research fields.

  In the following, an overview of the desirable properties of a \csm is given. Hereafter, \(C\), \(D\) and \(E\) are arbitrary \elh-concept descriptions and \(\sme\) denotes a \csm.

\subsection{Mathematics-related properties}

\paragraph{Symmetry.}
The \csm \(\sme\) is \emph{symmetric} if \(\sme(C,D) = \sme(D,C)\).
This mathematical property is widely recognised as an essential feature of a similarity measure for a \dl.
Indeed, it is the only property directly embedded in a \csm, in the Definition~\ref{dfn:sim}.
This necessity is not felt in other fields, more involved with the study of human reasoning, as seen in~\cite{Tve77}.

This property is one of the most widely recognised as an essential feature in similarity measures for \dl{}s, although it has been contested in fields which are more involved with the study of human reasoning, as seen in~\cite{Tve77}.
This is the only property that has been incorporated in the definition of a concept similarity measures in~\cite{LeTu12}.

\paragraph{Triangle inequality.}
The \csm \(\sme\) satisfies the \emph{triangle inequality} if \(1 + \sme(C,E) \ge \sme(C,D) + \sme(D,E)\) holds.
This is the other mathematical property which is inherited from the notion of a metric.
Of all the similarity measures investigated in~\cite{LeTu12}, the only one who fulfills the triangle inequality is the Jaccard Index, proposed in~\cite{Ja01} and adapted to the \dl \(\mathcal{L}_0\); however, this is not meant to be a \csm, to begin with.
Similarly to symmetry, the triangle inequality is  acknowledged as being not an essential feature of human reasoning in the work by Tversky in~\cite{Tve77}, while it is largely accepted in the context of description logics.

\subsection{\dl-related properties}

\paragraph{Equivalence closure.}
A \csm \(\sme\) is \emph{equivalence closed} whenever \(\sme(C,D) = 1\) is equivalent to \(C \equiv D\).
This property deeply relates with the semantical meaning of \elh-concept descriptions.

\paragraph{Equivalence invariance.}
A \csm \(\sme\) is \emph{equivalence invariant} if \(C \equiv D\) implies that \(\sme(C,E) = \sme(D,E)\)

Equivalence closure and equivalence invariance have found little to no criticism in the literature.
Indeed, it is regarded as necessary for two equivalent concepts to be totally similar, and equivalent concepts should share their similarity behaviour toward other concepts.
However, these properties have been not fulfilled in many concept similarity measures, defined and investigated, prior to the work done in~\cite{LeTu12}.

  \paragraph{Subsumption.}
  The \csm \(\sme\) is \emph{subsumption preserving} if \(C \subsume{K} D \subsume{K} E\) implies \(\sme(C,D)  \ge \sme(C,E)\); if \(C \subsume{K} D \subsume{K} E\) implies \(\sme(C,E) \le \sme(D,E)\), then it fulfills \emph{reverse subsumption preserving}.
  The notion of subsumption can be exploited to obtain an ordering over concepts.
  This, in turn, can be abstracted to the level of a concept similarity measure. How?
  Informally, if \(C\), \(D\) and \(E\) are \elh-concepts and \(C \subsume{K} D \subsume{K} E\), \(D\) is closer to \(E\) than \(C\), thus more similar;
  also, it can be noticed that \(D\) is closer to \(C\) than \(E\).
  These two insights can be captured as \emph{subsumption preserving} and \emph{reverse subsumption preserving}, where the concept similarity measure \(s\) obeys to the constraints \(s(C,E) \le s(D,E)\) (reverse subs.\ preserving) and \(s(C,D) \ge s(C,E)\) (direct subs.\ preserving).

  \paragraph{Structural dependence} is the last property that had been introduced for concept similarity measures.
  The idea behind structural dependence is that as the number of common \emph{features} of two concepts increases and the number of uncommon features is constant, then the similarity of these concepts must be increasing.
  Such distinct features can be thought of as sequence of atoms such that no two atoms in the sequence are in subsumption relation.
  Indeed, a \csm is \emph{structurally dependent} iff, given a sequence \({(C_k)}_{k=1}^n\) of atoms such that if \(1 \le i,j \le n\) and \(i \ne j\), then \(C_i \not\sqsubseteq C_j\), and defining \(D_n := \sqcap_{i=1}^n C_i \sqcap D\) for a given concept name, then \(\lim_{n \to \infty}s(D_n,E_n) = 1\) holds.

  \section{simi: a parametric concept similarity measure}
\label{sec:simi}

In this section, the concept similarity measure (\csm) \(\simi\) is introduced.
It belongs to the family of \emph{structural} measure, which are those measures defined according to the syntax of the concepts; they differ from \emph{interpretation-based} measures, where the definition relies on the interpretation of concepts (an example is the \emph{Semantic Similarity Measure} defined in~\cite{SemSim}).

  \paragraph{Preprocessing.}
  The usage of \simi requires a preliminar processing of the involved knowledge base.
  Since the input TBox \(\mathcal{T}\) is assumed to be unfoldable, it can be first normalised applying the rules enumerated in Definition~\ref{dfn:elh-nf} and then expanded, yielding a conservative extension \(\mathcal{T}^\star\) of the non-expanded version \(\mathcal{T}\), that only contains primitive concept names on the right hand side of concept definitions.
  Then, the input concepts are replaced by the expanded concepts, where only primitive concept names appear.
  The TBox is thus not necessary anymore, since its knowledge has been incorporated in the input concepts.
  Moreover, thanks to uniqueness (up to commutativity and associativity) of the \elh-normal form, \emph{equivalence invariance} is ensured for any measure employed to test concept similarity on the \kb.

  \paragraph{Directed and reverse similarity.}
  To test equivalence of two concepts \(C\) and \(D\), a possible way is to check whether subsumption holds in both directions, that is, \(C \sqsubseteq D\) and \(D \sqsubseteq C\).
  This approach can be astracted to the mathematical level, by first defining a \emph{directed similarity} measure --- a generalisation of the subsumption operator ---, to later combine values in both directions by means of another operator.
  The directed similarity measure \(\simi_d\) is going to be derived later in this section; finally, \(\simi\) is going to be defined as \(\simi(C,D) := \simi_d(C,D) \otimes \simi_d(D,C)\).
  The chosen operator should fit certain conditions, which guarantee that the resulting measure complies with the desired properties.
  \begin{definition}[Fuzzy connector]
    A \emph{fuzzy connector} \(\otimes \colon {[0,1]}^2 \to [0,1]\) is an operator that fulfills the following conditions, for all \(x, y \in [0,1]\):
    \begin{itemize}
      \item \(x \otimes y = y \otimes x\) --- \emph{commutativity};
      \item \(x \otimes y = 1\) iff \(x = y = 1\) --- \emph{equivalence closure};
      \item if \(x \le y\), then \(1 \otimes x \le 1 \otimes y\) --- \emph{weak monotonicity};
      \item if \(x \otimes y = 0\), then at least one between \(x \ne 0\) or \(y \ne 0\) does not hold.
    \end{itemize}
  \end{definition}

  \begin{example}
    The operator \(\otimes_{ex} \colon {[0,1]}^2 \to [0,1]\) defined as
    \[
    x \otimes_{ex} y := \frac{x^2y + y^2x}{2} \qquad x,y \in [0,1]
    \]
    is a fuzzy operator.
    The commutativity follows from the commutativity of the sum and the multiplication over \([0,1]\). Weak monotonicity holds, because \(x \le y\) implies that \((x \cdot (x+1))/2 \le (y \cdot (y+1))/2\).
    This operator is equivalence closed: assume that \(x \otimes_{ex} y = 1\) and \(x \ne 1\), \(y \ne 1\), and \(x \le y\) without loss of generality.
    Since \(x^2y = 2 - xy^2\) and \(x^2y \le xy^2\), it follows that \(2-xy^2 \le xy^2\), hence \(xy^2 \ge 1\); however, this is a contradiction to the hypothesis that \(x\), \(y \in [0,1]\), thus at least one between \(x\), \(y\) equals \(1\).
    Suppose that \(y = 1\): then, from \((x^2 + x)/2 = 1\) follows that \(x = 1\); analogously, it can be proved that \(x = 1\) implies \(y = 1\).
  Finally, from \((x^2y + y^2x)/2 = 0\) it follows that \(x^2y = - y^2x\); then, one can conclude that \(x^2y = y^2x = 0\), hence \(x = 0\) or \(y = 0\).
\end{example}
  
In the \(\simi{}\) framework, the choice of a fuzzy operator provides \emph{symmetry} --- thanks to commutativity ---, \emph{equivalence closure} and \emph{sumbsumption preserving}, which follows from weak monotonicity.

\paragraph{The need of a primitive measure.}
The inspiration for \(\simi_d\) comes from the Jaccard Index, described in~\cite{Ja01};
indeed, the starting point of its derivation is the function
\begin{equation}
  \label{eqn:d}
  d(C,D) :=
\frac{\abs{\atom{C} \cap \atom{D}}}{\abs{\atom{C}}}.
\end{equation}
The definition given in~\eqref{eqn:d} can be rephrased to take into account the way \(C\) and \(D\) are built as concept descriptions, by making use of their atoms:
\begin{equation}
  \label{simid:rewr}
  \abs{\atom{C} \cap \atom{D}} := \sum_{C^\prime \in \atom{C}}\max_{D^\prime \in \atom{D}}
                                  \delta_{C^\prime}^{D^\prime},
\end{equation}
where \(\delta_{C^\prime}^{D^\prime}\) is \(1\) when \(C^\prime = D^\prime\), \(0\) otherwise.
Several issues are now rising, with the expression given in~\eqref{simid:rewr}:
existential restrictions are dealt in a poor way and, in general, this formalisation makes it impossible to distinguish the degree of similarity of two concept descriptions, since \(\delta_{C^\prime}^{D^\prime}\) checks syntactical equality of two concepts, assuming that otherwise, they cannot be similar at all.

To solve these problems, the function \(\delta\) is generalised to a function over concept names and role names, with adequate conditions, that will guarantee some desired properties for \(\simi\).
\begin{definition}[Primitive measure]
  \label{dfn:pm}
  A function \(pm \colon \mathbf{C}^2 \cup \mathbf{R}^2 \to [0,1]\) is a \emph{primitive measure} if:
  \begin{enumerate}
    \item for all concept names \(A\), \(B\), \(pm(A,B) = 1\) iff \(A = B\);
    \item for given role names \(r\), \(s\), \(t\):
    \begin{enumerate}
      \item \(pm(r,s) = 1\) iff \(s \sqsubseteq r\),
      \item if \(r \subsume{R} s\), then \(pm(r,s) > 0\),
      \item if \(t \subsume{R} s\), then \(pm(r,s) \le pm(r,t)\).
    \end{enumerate}
  \end{enumerate}
\end{definition}
The Definition~\ref{dfn:pm} is sufficient to ensure that \(\simi\) will be \emph{equivalence closed} and \emph{subsumption preserving}: indeed, the properties of \(pm\) to prove that these properties of \(\simi\) hold for ``atomic'' concepts, in particular, can be used in the proof of Lemma~\ref{lem:simid}.

Coming back to the similarity function \(d\) defined in~\eqref{eqn:d}, it is now possible to deal with existential restrictions.
This is done, according to the following schema:

\begin{enumerate}
  \item When similarity between concept names is computed, the primitive measure \(pm\) is sufficient, since it handles this case directly.
  \item If a concept name and an existential restriction are compared, the similarity is \(0\).
  \item In the final case, it would be needed two compute the similarity of two existential restrictions, namely \(\exists{}r.C^\star{}\) and \(\exists{}s.D^\star{}\).
  To achieve this, both role names \(r,s\) and inner concepts \(C^\star,D^\star\) should be considered.
  While the roles are handled by \(pm\), a recursive call of the measure over \(C^\star\), \(D^\star\) is needed.
  To combine these two values, and to establish how much influence does each one exert on the final measure, an affine combination is introduced, along with a value \(w \in (0,1)\):
  \[
  d(\exists{}r.C^\star,\exists{}s.D^\star) :=
  pm(r,s) \cdot \left(w + (1-w)d(C^\star,D^\star)\right).
  \]
\end{enumerate}
  In this way, the case \(d(C^\star,D^\star) = 0\) allows one to understand whether the roles are similar --- then the similarity equals \(w\), otherwise it would be equal to \(0\).
  What is a \emph{good} value for \(w\)? By default, \(w = 0.01\).
  A feasible value for \(w\) may be proportionally decreasing with respect to the natural number \(n\) such that the two concepts \(C\), \(D\) of the form
  \[
  C = \overbrace{\exists{}r.\dotsb{}\exists{}r.}^{n\;\text{times}}A \quad
  D = \overbrace{\exists{}r.\dotsb{}\exists{}r.}^{n\;\text{times}}B
  \]
  with \(pm(A,B) = 0\) are almost, if not totally similar.

  \paragraph{Information on similarity.}
  The last issue that was dealt with in~\cite{LeTu12} concerned the \(\mathbf{\max}\) operator appearing in the definition of the similarity measure developed so far.
  Indeed, the claim is that such an operator is not making full use of the available information, to compute the similarity of two concepts; that happens, because only the \emph{best matching} atom in the set \(\widehat{D}\) is kept, discarding all the others, which might still be similar to the considered atom of \(\widehat{C}\).

  The maximum operator fits into a class of functions called \emph{triangular conorms}; %, presented in \todo{Insert reference here.[14] on paper.};
  in particular, it is an instance of a \emph{bounded \(t\)-conorm}.
  \begin{definition}%[Bounded t-conorm]
    An operator \(\oplus \colon {[0,1]}^2 \to [0,1]\) is a \emph{bounded t-conorm} if it satisfies the following conditions: \(\oplus\) is commutative, associative, \(0\) is the neutral element and for all \(x\), \(y\), \(z\), \(w \in [0,1]\),
    \begin{itemize}
      \item if \(x \le z\) and \(y \le w\), then \(x \oplus y \le z \oplus w\), i.e. \(\oplus{}\) is \emph{monotonic};
      \item if \(x \oplus y = 1\), then \(x = 1\) or \(y = 1\), i.e. \(\oplus{}\) is \emph{bounded}.
    \end{itemize}
  \end{definition}
  Moreover, \(\mathbf{\max}\) is the strongest \(t\)-conorm, which means that all the other \(t\)--conorms yield a value equal or greater than \(\mathbf{\max}\); this, intuitively, reflects the fact that \(\mathbf{\max}\) does not make good use of the available information.
  Finally, \(0\) is the unit element of \(t\)-conorms: this implies that totally dissimilar concepts are not influencing the final value of the measure.
  Thus, in the final definition of \(\simi_d\), \(\mathbf{\max}\) will be replaced by a \emph{bounded \(t\)-conorm}, which is a parameter of the framework.

  \paragraph{Relevance of concept names.}
  Up until now, all the atoms of \(\mathcal{C}\)(\elh) had the same relevance, in determining how concepts differ from each other.
  In practical application, however, it is often the case that different concepts have different relevance: an example in the medical context is that of operating at the cardiovascular level; in this case, concepts related to heart are more relevant then ones related to brain.

  Hence, another parameter that has been added to the framework is a function that specifies how much an atom influences the similarity between concepts.
  \begin{definition}%[Weighting function]
    A \emph{weighting function} is a mapping \(g \colon \mathbf{A} \to \mathbb{R}_+\), where \(\mathbf{A}\) is the set of \emph{atoms}.
  \end{definition}

  The weighting function \(g\) can be seen as a generalization of the cardinality of the set of atoms, introduced in the equation~\eqref{eqn:d}.
  Indeed, \(g(\cdot) = 1\) would bring cardinality back into the definition of \(\simi_d\) that is given in the following section.
  % Since \(\mathbf{A}\) can contain infinitely many elements, it would be desirable to find a way to define a weighting function by using only some "essential" information.

  % \paragraph{Primitive weighting.}
  % Thanks to the preprocessing phase, only primitive concept names and role names appear in the input concepts.
  % In particular, it is possible to reduce each concept in \elh-normal form that does not contain any instance of \(\top\): hence, it is not needed to weigh it.
  % Another aspect of this normal form is that it is possible to weigh just a subset of \(\mathbf{R}\), that is the image of the function \(ch\) used to compute the \elh-normal form; in this way, at most \(\lvert \mathbf{R} \rvert\) roles would need to be weighted.
  % In this case, all the role names are used.
  % Finally, concept names can be either primitive or defined:
  % in the latter case, thanks to preprocessing again, a unique definition can be find in the expanded TBox, which right side contains exclusively primitive concept names.

  % Using these facts, the following definition is introduced.
  % \begin{definition}[Primitive weighting function]
  %   A \emph{primitive weighting function} is a mapping \(f \colon \mathbf{P} \cup \mathbf{R} \to \mathbb{R}_+\), where \(\mathbf{P}\) is the subset of \(\mathbf{C}\) of \emph{primitive} concept names.
  % \end{definition}
  % Given a weighting primitive function, it is possible to extend it to a weighting function \(g\) in the following way:
  % \begin{equation}\label{prim-weigh}
  %   g(C) :=
  %   \begin{cases}
  %     f(C) & C \in \mathbf{P} \\
  %     f(r) & C = \exists{}r.D, D \in \mathcal{C}(\mathcal{ELH}) \\
  %     \max_{D^\prime \in \widehat{D}}f(D) & C \equiv D \; \text{appears in the expanded TBox.}
  %   \end{cases}
  % \end{equation}
  % \begin{proposition}
  %   With the expansion given in~\eqref{prim-weigh}, every primitive weighting function uniquely determines a weighting function.
  % \end{proposition}
  % \begin{proof}
  %   Assume that the weighting functions \(g\), \(h\) are both derived from \(f\) as in~\eqref{prim-weigh}.
  %   \begin{enumerate}
  %     \item If \(C \in \mathbf{P}\), then
  %     \(g(C) = f(C) = h(C)\) by definition;
  %     \item if \(C =  \exists{}r.D, D \in \mathcal{C}(\mathcal{ELH})\), then \(g(C) = f(r) = h(C)\);
  %     \item if \(C \equiv D\) is the \emph{unique} definition of \(C\) in the TBox \(\mathcal{T}\), then
  %     \(g(C) = \max_{D^\prime \in \widehat{D}}f(D) = h(C)\).
  %   \end{enumerate}
  %   Thus, it is possible to conclude that \(g(C) = h(C)\) for all \(C \in \mathbf{A} \setminus \lbrace \top \rbrace\).
  % \end{proof}
  % This approach can be further refined, for example, by substituting the \(\max\) function with another, perhaps more suitable function.
  % %Notice that, given a primitive weighting function, this extension to a weighting function can be computed in polynomial time (this last claim needs to be verified).

\subsection{The measure of directed similarity: \(\simi_d\)}

The final step of the derivation is presented in this section. A measure of \emph{directed similarity}, namely \(\simi_d\), is introduced, culminating the process started by using a variation of the Jaccard Index.

  \begin{definition}[\(\simi_d\)]\label{simi-d}
    The measure of \emph{directed similarity} \(\simi_d \colon {\mathcal{C}(\mathcal{ELH})}^2 \to [0,1]\), given the following parameters:
    \begin{enumerate}
      \item a bounded \(t\)-conorm \(\oplus\),
      \item a primitive measure \(pm\),
      \item a weighting function \(g\),
      \item \(w \in (0,1)\),
    \end{enumerate}
    is defined as:
    \begin{align}
      \simi_d(\top,D) &= 1 \quad
      \text{for all}\; D \in \mathcal{C}(\mathcal{ELH}) \\
      \intertext{which captures the fact that \(D \subsume{} \top\) for all \elh-concepts;}
      \simi_d(A,B) &= pm(A,B) \quad A,B \in \Conc \\
      \simi_d(\exists{}r.C,\exists{}s.D) &=
      pm(r,s) \cdot \left(w + (1-w)\simi_d(C,D)\right) \\
      \intertext{which are the cases dealing with atomic concepts, and for the general case:}
      \simi_d(C,D) &=
      \frac{\sum_{C^\prime \in \widehat{C}}\left(g(C^\prime) \cdot \bigoplus_{D^\prime \in \widehat{D}} \simi_d(C^\prime, D^\prime)\right)}%
      {\sum_{C^\prime \in \widehat{C}}g(C^\prime)} \quad \text{if}\; C \ne \top \label{simi:general}\\
      \simi_d(C,D) &= 0 \quad \text{otherwise}.
    \end{align}
  \end{definition}

  As previously stated, the definition of the \csm \simi relies on connecting two values of \(\simi_d\), by means of a fuzzy connector. This terminates the process of derivation of \(\simi\).

  \begin{definition}[\(\simi\)]
    The similarity measure \(\simi\) is defined as \(\simi(C,D) := \simi_d(C,D) \otimes \simi_d(D,C)\), where \(\otimes{}\) is a fuzzy connector.
  \end{definition}

  Up until now, many properties for the parameters have been specified, without explicitly using them in ensuring properties for \simi.

  The following lemma shows an interesting property of \(\simi_d\), which can be later used to show some useful features of \simi.

  \begin{lemma}
    \label{lem:simid}
    Let \(C\), \(D\) and \(E\) be \elh-concept descriptions.
    Then,
    \begin{enumerate}
      \item \(\simi_d(C,D) = 1\) iff \(D \sqsubseteq C\);
      \item if \(D \sqsubseteq E\), then \(\simi_d(C,E) \le \simi_d(C,D)\).
    \end{enumerate}
  \end{lemma}
  \begin{proof}
    The proof of this theorem can be found in~\cite{LeTu12}.
  \end{proof}

  The following results, which make use of what introduced and defined so far, state that the defined \csm achieves a large number of desirable properties for similarity measures, including the ones that motivated this work in first place: equivalence invariance and equivalence closure.

  \begin{theorem}
    The \csm \(\simi\) is symmetric, closed and invariant under equivalence and preserves subsumption, for each instantiation of its parameters.
  \end{theorem}
  \begin{proof}
    Let \(C\), \(D\) and \(E\) be \elh-concept descriptions.

    The symmetry of \(\simi\) follows from the commutativity of the fuzzy connector.

    The invariance under equivalence of \(\simi\) is guaranteed by the preprocessing step.
    Indeed, two semantically equivalent \elh-concepts are transformed into the same \elh-normal form.
    By assuming that \(C \equiv D\), this insight allows the conclusion \(\simi(C,E) = \simi(D,E)\).

    The closure under equivalence is proved by the following equivalence chain:
    \begin{equation*}
      \begin{split}
        C \equiv D &\iff C \sqsubseteq D \land D \sqsubseteq C \iff
        \simi_d(C,D) = 1 \land \simi_d(D,C) = 1 \\ &\iff
        \simi(C,D) = \simi_d(C,D) \otimes \simi_d(D,C) = 1.
      \end{split}
    \end{equation*}

    In a similar way, subsumption preserving can be proved.
  \end{proof}
  The \csm \(\simi\) fulfills most of the desired properties.
  Indeed, in~\cite{LeTu12}, it is shown that it is \emph{equivalence closed}, \emph{equivalence invariant}, \emph{symmetric} and \emph{subsumption preserving} for every choice of the parameters.
  For adequates choice of the fuzzy connector and the weighting function, \emph{structural dependence} can be achieved.
  The only properties that cannot be guaranteed by \(\simi\) are \emph{triangle inequality} and \emph{reverse subsumption preserving}, since similarity values between the atoms of the involved concepts are not taken into account, in the computation of the similarity value of the concepts.

  Finally, it is shown how \simi can be computed efficiently, under certain assumptions.

  \begin{proposition}
    The \csm \(\simi\) can be computed in time polynomial in the size of the measured concepts, assuming that the provided fuzzy connector, bounded t-conorm, primitive measure and weighting function can be computed in polynomial time. 
  \end{proposition}
  \begin{proof}
    From the assumption that the fuzzy connector can be computed in polynomial time, it suffices to show that \(\simi_d\) can be computed in polynomial time.
    The assumption that the primitive measure can be computed in polynomial time allows to conclude that all the cases of \(\simi\) where \emph{atoms} are measured run in polynomial time.

    In the most general case~\eqref{simi:general}, the number of similarity values that have to be computed amounts to \(\lvert \widehat{C} \rvert \cdot \lvert \widehat{D} \rvert\).
    The depth of the recursion call to \simi, in the case where \(\exists{}r.E\) and \(\exists{}s.F\) are atoms belonging to \(\atom{C}\) and \(\atom{D}\) respectively, is bounded by the size of the concepts \(C\) and \(D\).
    Thus, we can conclude that \(\simi\) runs overall in polynomial time, in the size of the measured concepts.
  \end{proof}

  \section{Related work}

\paragraph{Improving performances.}
Another \csm for \elh has been investigated in~\cite{simEL}, that outperforms \(\simi\), while maintaining the same similarity properties.
The approach adopted by the authors involves the usage of the so-called \emph{\elh-concept description trees} as a preprocessing step, and its motivated by the homomorphism-based structural subsumption characterisation.
Despite looking similar to \(\simi\) in its definition, this \csm takes a bottom-up approach, which allows for rejection of exceeding recursive calls and the reuse of partial solutions to subproblems.
\(\simi\), with its top-down approach, can potentially have an increasing amount of repeated subroutines which are avoided by this new measure.

\paragraph{The triangle inequality.}
Among the desired properties for a \csm, triangle inequality is the least achieved.
Can this be obtained, without sacrificing other, more relevant properties?
In~\cite{TriEq}, the translation of \el-concepts to description trees is also employed. Then, a dissimilarity measure based on \emph{concept relaxations} is defined. A \emph{concept relaxation} can be intuitively understood as a stepwise generalization of a concept.
This new dissimilarity measure respects the triangle inequality and fulfills all the other desired properties, except for structural dependency.

  \paragraph{General TBoxes.} A general treatment of TBoxes in the context of similarity evaluation of relaxed instances of \el-concepts, had been included in a follow-up work, which introduces a treatment for both the general and the unfoldable case (\cite{Ec14}).

  \paragraph{Approximation.} Approximation grades, induced by concept similarity measures and complexity of relative threshold \dl{}s, are discussed in~\cite{Ba17}, where it is shown how \csm{}s for \el description logics can be employed to derive a grade membership function and consequently a threshold \dl.

  \section{Conclusion}

  In this report, the concept similarity measure \(\simi\), which possess some interesting and useful properties, has been introduced.

  Some of its properties have been explored, such as invariance and closure under equivalence or symmetry, and it has been shown how its computation can be considered efficient.

  The path towards the definition of such a measure has been illustrated, with a focus on each single parameter, along with the motivations that justified their usage.
  Most of the desired properties have been successfully incorporated, while others, like triangle inequality, have been not included, but may be possibly achieved.

  While \elh is a good and widely used \dl, the question is whether a similar measure might be extended to more expressive description logics, without losing its peculiar features.
  It would also be worthy to investigate the possibility of using general TBoxes, where cycles may break the behaviour of \(\simi\) as it is defined now.

  Finally, it would be interesting to obtain a full implementation of \(\simi\), in order to be able to assess it against other existing measures, as well as to start using it to evaluate concept similarity in ontologies.

  % \begin{thebibliography}{1}
  %   \bibitem{Lae15}
  %     Lätsch, Paul and Turhan, Anni-Yasmin.
  %     \textit{Simi-Framework for Concept-Similarity-Measures: property analysis and expansion to more powerful Description Logics}.
  %     Diplomarbeit, Technische Universität Dresden, 2015.
  % \end{thebibliography}
  \bibliography{seminar_further-DL}%

\end{document}
