\section{The \(\mathcal{ELH}\) Description Logic}

Hereafter, the reader must assume that the \elh description logic (\dl) is used where not differently specified.
For a detailed treatment of the notions here introduced, the interested reader can go to~\cite{DLbook}.

The syntax of a \dl makes use of two finite and disjoint sets of \emph{concept names}, \(\Conc\) and \emph{role names}, \(\R\) to build \emph{concept descriptions}, by means of \emph{concept constructors}.
The semantics of such descriptions is defined with respect to an interpretation \(\I = (\DeltaI,\dotI)\), where \(\DeltaI \ne \emptyset\) is the \emph{interpretation domain} and \(\dotI\) is a mapping which assigns a subset of \(\DeltaI{}\) to each concept and a binary relation over \(\DeltaI\) to each role.
The syntax and the semantics of the elements of the set of \emph{\elh-concept descriptions} \(\celh\) are introduced in~\cref{tbl:el}.

\begin{table}[t]
  \caption{Constructors of the \dl \el, along with their syntax and semantics w.r.t. an interpretation \(\I = (\DeltaI,\dotI)\). Here, \(A \in \Conc\), \(r \in \R\) and \(C, D \in \celh\).}
  \label{tbl:el}
  \centering
  \begin{tabular}{lcc}
    \toprule
    Name & Syntax & Semantics \\
    \midrule
    Top & \(\top\) & \(\DeltaI\) \\
    Concept name & \(A\) & \(A^\I\) \\
    Conjunction & \(C \sqcap D\) & \(C^\I \cap D^\I\)\\
    Existential restriction & \(\exists{}r.C\) &
    \(\lbrace d \in \DeltaI \mid \exists e \in \DeltaI.
    (d,e) \in r^\I \land e \in C^\I \rbrace\) \\
    \bottomrule
  \end{tabular}
\end{table}

In \elh, each concept description can be expressed in conjunctive form.
This feature will be later exploited to compute the value of \simi by means of the so-called \emph{atoms} of the concepts there compared.
\begin{definition}[Atoms]
  If \(\Conc\) and \(\R\) are the set of concept and role names (resp.), the set \(\A\) of \emph{\elh-atoms} (or \emph{atoms}) is defined as
  \begin{equation}
    \A := \Conc \cup \lbrace \top \rbrace \cup \lbrace \exists{}r.D \mid r \in \R, D \in \mathcal{C}(\mathcal{ELH}) \rbrace.
  \end{equation}
  Moreover, \(\widehat{\cdot} \colon \celh \to 2^{\celh}\) is defined as \(\widehat{C} := \lbrace C_i\rbrace_{i=1}^n\), where \(C = \bigsqcap_{i=1}^n C_i \in \celh\).
\end{definition}

The \kb adopted hereafter has two components: the \emph{terminological} knowledge and another components, expressing information about the \emph{role hierarchy}.
These are represented by finite sets of \emph{axioms}: the terminological knowledge with  a \emph{TBox} \(\mathcal{T}\) and the knowledge about role hierarchy in an \emph{RBox} \(\mathcal{R}\). The syntax and the semantics of the elements of these sets are defined in~\cref{tbl:boxes}.
A concept name (resp. role name) that does not appear on the left-hand side (\lhs) of any axiom in a TBox (resp. RBox) is called \emph{primitive}.

\begin{table}[b]
  \caption{Types of axioms that are found in a TBox or in an RBox, with their syntax and semantics w.r.t. an interpretation \(\I\). Here, \(A \in \Conc\), \(r,s \in \R\) and \(C, D \in \celh\).}
  \label{tbl:boxes}
  \centering
  \begin{tabular}{llcc}
    \toprule
    Family & Name & Syntax & Semantics \\
    \midrule
    TBox & \gci & \(C \sqsubseteq D\) &\(C^\I \subseteq D^\I\) \\
    & Primitive concept definition & \(A \subsume{} C\) & \(A^\I \subseteq C^\I\) \\
    & Concept definition & \(A \equiv C\) & \(A^\I = C^\I\) \\
    \midrule
    RBox & \ria & \(r \sqsubseteq s\) & \(r^\I \subseteq s^\I\) \\
    \bottomrule
  \end{tabular}
\end{table}

As an assumption that will make the definition of \(\simi\) easier, only a restricted class of TBoxes will be considered, namely, the \emph{unfoldable} TBoxes.
\begin{definition}[Unfoldable TBox]
  A TBox \(\mathcal{T}\) is called \emph{unfoldable} iff it is acyclic, consists of concept definitions and primitive concept definitions only and every concept name occurs at most once on a \lhs of a concept axiom.    
\end{definition}
An unfoldable TBox \(\mathcal{T}\) can be expanded in a TBox \(\mathcal{T}^\star\) which is semantical equivalent to it, contains (primitive) concept definitions, where only primitive concept names appear on the right-hand side of the concept definitions.
Additionally, primitive concept definitions can be transformed into concept definitions, adding new concept names to the TBox.
In the definition of \(\simi\), the input concepts will be expanded by replacing each defined concept with its definition in the TBox; consequently, only primitive concept names will appear and the knowledge contained in the TBox will not be needed.

An interpretation \(\I\) is a \emph{model} for a TBox \(\mathcal{T}\) if for all the axioms \(C \subsume{} D \in \mathcal{T}\) (resp. \(C \equiv D\)), \(C^\I \subseteq D^\I\) holds (resp. \(C^\I = D^\I\)).
Similarly, \(\I\) is a model for an RBox \(\mathcal{R}\), if \(r^\I \subseteq s^\I\) holds for all \(r \subsume{} s \in \mathcal{R}\).

Among the various reasoning services that can be applied over \elh \kb{}s, a commonly used one is the \emph{subsumption} of concepts: a concept \(C\) is \emph{subsumed} by \(D\) w.r.t.\ a \kb \(\mathcal{K}\) iff \(C^\I \subseteq D^\I\) for all models \(\I\) of \(\mathcal{K}\), in symbols \(C \subsume{K} D\); if \(C\) and \(D\) both subsume each other, then they are \emph{equivalent}, in symbols \(C \equiv_{\mathcal{K}} D\).
Similarly, \(r \subsume{K} s\) if \(r^\I \subseteq s^\I\) for all models \(\I\) of \(\mathcal{R}\).

A normal form for \el concepts is well described in~\cite{DLbook}; in~\cite{LeTu12}, this is extended to take into account the content of the RBox.
Such a normal form is computable in polynomial time.
Specifically, a choice function \(ch{}\) is devised such that every role is replaced by a picked representative of its \(\equiv_\mathcal{R}\)-equivalence class.
\begin{definition}[\elh-normal form]
  \label{dfn:elh-nf}
  A concept \(C \in \mathcal{C}\)(\elh) is in \emph{\elh-normal form} w.r.t. a RBox \(\mathcal{R}\) if all its subconcepts are closed under the application of the following rules:
  \begin{align*}
    A \sqcap \top &\mapsto A & A \sqcap A &\mapsto A \tag{1--2}\\
    \exists{}r.C^\prime &\mapsto \exists{}ch([r]).C^\prime &
    \exists{}r.C^\prime \sqcap \exists{}s.D^\prime &\mapsto
    \exists{}r.C^\prime\,\,\text{if}\,\,r\subsume{R}s, C^\prime \sqsubseteq D^\prime \tag{3--4}
  \end{align*}
\end{definition}