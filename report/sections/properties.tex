\section{Properties of similarity measures}
\label{sec:properties}
How may one quantitatively measure the amount of similarity between two concepts?
In many real-world examples, ranging from cognitive sciences, biology, physics, to computer vision, the underlying mathematical structure is that of a \emph{metric space}, which is endowed with a \emph{metric}.
In particular, the similarity of two concepts can be interpreted as a relative measure: thus, it is possible to take normalised metric spaces, where distances are bound to live in the unit interval.
In this view, the metric distance between two concepts expresses their dissimilarity: thus, two totally similar concepts will be in the same position in the space, with dissimilar concepts being further and further.
Therefore, given a normalised metric space \((D,d)\), the similarity function \(s\) inducted by \(d\) corresponds to \(s(a,b) := 1 - d(a,b)\).

\begin{definition}[(concept) similarity function]
  \label{dfn:sim}
  A \emph{similarity function} \(s \colon D^2 \to [0,1]\) satisfies these conditions, for all \(a,b,c \in D\):
  \begin{enumerate}
    \item \(s(a,b) = 1\) iff \(a = b\),
    \item \(s(a,b) = s(b,a)\),
    \item \(1 + s(a,b) \ge s(a,c) + s(c,b)\).
  \end{enumerate}
  A \emph{concept similarity measure} (\csm) over \elh-concepts is defined as a symmetric mapping \(\sme \colon \celh^2 \to [0,1]\).
\end{definition}

  How much can be directly embedded in this definition?
  Which are the properties globally recognised as peculiar to measure concept similarity?
  A \emph{property}, here, is seen as a formalisation of the expected behaviour of the defined measure.
  Unsurprisingly, almost all the features, that one can think of as desirable for such a tool, have encountered some criticism in different research fields.
  In the rest of the section, \(C\), \(D\) and \(E\) are arbitrary \elh-concept descriptions and \(\sme\) denotes a \csm.

\subsection{Mathematics-related properties}

\paragraph{Symmetry.}
The \csm \(\sme\) is \emph{symmetric} if \(\sme(C,D) = \sme(D,C)\).
This mathematical property is widely recognised as an essential feature of a similarity measure for a \dl.
Indeed, it is the only property directly embedded in a \csm, in the Definition~\ref{dfn:sim}.
This necessity is not felt in other fields, more involved with the study of human reasoning, as seen in~\cite{Tve77}.

This property is one of the most widely recognised as an essential feature in similarity measures for \dl{}s, although it has been contested in fields which are more involved with the study of human reasoning, as seen in~\cite{Tve77}.
This is the only property that has been incorporated in the definition of a concept similarity measures in~\cite{LeTu12}.

\paragraph{Triangle inequality.}
The \csm \(\sme\) satisfies the \emph{triangle inequality} if \(1 + \sme(C,E) \ge \sme(C,D) + \sme(D,E)\) holds.
This is the other mathematical property which is inherited from the notion of a metric.
Of all the similarity measures investigated in~\cite{LeTu12}, the only one who fulfills the triangle inequality is the Jaccard Index, proposed in~\cite{Ja01} and adapted to the \dl \(\mathcal{L}_0\); however, this is not meant to be a \csm, to begin with.
Similarly to symmetry, this is  acknowledged as being not an essential feature of human reasoning in the work by Tversky in~\cite{Tve77}, while it is largely accepted in the context of description logics.
The concept similarity measure \(\simi\), as defined later in this report, does not fulfill this condition.\todo{Move this consideration further.}

\subsection{\dl-related properties}

\paragraph{Equivalence closure.}
A \csm \(\sme\) is \emph{equivalence closed} whenever \(\sme(C,D) = 1\) is equivalent to \(C \equiv D\).
This property deeply relates with the semantical meaning of \elh-concept descriptions.

\paragraph{Concept equivalence.} Equivalence closure and equivalence invariance have found little to no criticism in the literature.
Indeed, it is regarded as necessary for two equivalent concepts to be totally similar, and equivalent concepts should share their similarity behaviour toward other concepts.
These properties can be formally stated as follows: for all concepts \(C\), \(D\) and \(E\), a concept similarity measure \(s\) is \emph{equivalence invariant} if \(C \equiv D\) implies that \(s(C,E) = s(D,E)\), while it is \emph{equivalence closed} if \(s(C,D) = 1\) is equivalent to \(C \equiv D\).

  \paragraph{Subsumption.}
  The \csm \(\sme\) is \emph{subsumption preserving} if \(C \subsume{K} D \subsume{K} E\) implies \(\sme(C,D)  \ge \sme(C,E)\); if \(C \subsume{K} D \subsume{K} E\) implies \(\sme(C,E) \le \sme(D,E)\), then it fulfills \emph{reverse subsumption preserving}.
  The notion of subsumption can be exploited to obtain an ordering over concepts.
  This, in turn, can be abstracted to the level of a concept similarity measure. How?
  Informally, if \(C\), \(D\) and \(E\) are \elh-concepts and \(C \subsume{K} D \subsume{K} E\), \(D\) is closer to \(E\) than \(C\), thus more similar;
  also, it can be noticed that \(D\) is closer to \(C\) than \(E\).
  These two insights can be captured as \emph{subsumption preserving} and \emph{reverse subsumption preserving}, where the concept similarity measure \(s\) obeys to the constraints \(s(C,E) \le s(D,E)\) (reverse subs.\ preserving) and \(s(C,D) \ge s(C,E)\) (direct subs.\ preserving).

  \paragraph{Structural dependence} is the last property that had been introduced for concept similarity measures.
  The idea behind structural dependence is that as the number of common \emph{features} of two concepts increases and the number of uncommon features is constant, then the similarity of these concepts must be increasing.
  Such distinct features can be thought of as sequence of atoms such that no two atoms in the sequence are in subsumption relation.
  Indeed, a \csm is \emph{structurally dependent} iff, given a sequence \({(C_k)}_{k=1}^n\) of atoms such that if \(1 \le i,j \le n\) and \(i \ne j\), then \(C_i \not\sqsubseteq C_j\), and defining \(D_n := \sqcap_{i=1}^n C_i \sqcap D\) for a given concept name, then \(\lim_{n \to \infty}s(D_n,E_n) = 1\) holds.
