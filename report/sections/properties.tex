\section{Measuring concept similarity}

  How may one quantitatively measure the amount of similarity between two concepts?
  In many real-world examples, ranging from cognitive sciences, biology, physics, to computer vision, the underlying mathematical structure is that of a \emph{metric space}, which is endowed with a \emph{metric}.
  In particular, the similarity of two concepts can be interpreted as a relative measure: thus, it is possible to take normalised metric spaces, where distances are bound to live in the unit interval.
  In this view, the metric distance between two concepts expresses their dissimilarity: thus, two totally similar concepts will be in the same position in the space, with dissimilar concepts being further and further.
  Therefore, given a normalised metric space \((D,d)\), the similarity function \(s\) inducted by \(d\) corresponds to \(s(a,b) := 1 - d(a,b)\).
  Generally speaking, a \emph{similarity function} \(s \colon D^2 \to [0,1]\) comply with following specification, where \(a,b,c \in D\):
  \begin{enumerate}
    \item \(s(a,b) = 1\) iff \(a = b\),
    \item \(s(a,b) = s(b,a)\),
    \item \(1 + s(a,b) \ge s(a,c) + s(c,b)\).
  \end{enumerate}

  Thus, a \emph{concept similarity measure} over \elh-concepts consists of a symmetric mapping \(\mathrm{sim} \colon {\mathcal{C}(\mathcal{ELH})}^2 \to [0,1]\).

  How much can be directly embedded in this definition?
  Which are the properties globally recognised as peculiar to measure concept similarity?
  A \emph{property}, here, is seen as a formalisation of the expected behaviour of the defined measure.
  Unsurprisingly, almost all the features, that one can think of as desirable for such a tool, have encountered some criticism in different research fields.

  \paragraph{Symmetry.} This property is one of the most widely recognised as an essential feature in similarity measures for \dl{}s, although it has been contested in fields which are more involved with the study of human reasoning, as seen in~\cite{Tve77}.
  This is the only property that has been incorporated in the definition of a concept similarity measures in~\cite{LeTu12}.
  % Indeed, we extend what said so far, by calling \emph{concept similarity measure} a symmetric mapping \(\mathrm{sim} \colon {\mathcal{C}(\mathcal{ELH})}^2 \to [0,1]\), i.e. \(\mathrm{sim}(C,D) = \mathrm{sim}(D,C)\) for all \elh-concepts \(C\), \(D\).

  \paragraph{Triangle inequality.} This is similarly acknowledged as being not an essential feature of human reasoning in the work by Tversky in~\cite{Tve77}, while it is largely accepted in the context of description logics.
  A context similarity measure \(s\) over the set \(D\) satisfies the triangle inequality if, for all \(a\), \(b\), \(c \in D\),
  \(s(a,c) \le s(a,b) + s(b,c)\).
  Strangely enough, almost none of the concept similarity measures found in the literature up until 2012 fulfills fulfills the triangle inequality.
  The only exception is the Jaccard Index, which is \emph{not} meant to be such a measure.
  The concept similarity measure \(\simi\), as defined later in this report, does not fulfill this condition.

  Some advances in the direction of defining a measure with this property can be found in~\cite{DAB14}, where a measure respecting the triangle inequality (but missing other properties) had been provided for the \el \dl.

  \paragraph{Concept equivalence.} Equivalence closure and equivalence invariance have found little to no criticism in the literature.
  Indeed, it is regarded as necessary for two equivalent concepts to be totally similar, and equivalent concepts should share their similarity behaviour toward other concepts.
  These properties can be formally stated as follows: for all concepts \(C\), \(D\) and \(E\), a concept similarity measure \(s\) is \emph{equivalence invariant} if \(C \equiv D\) implies that \(s(C,E) = s(D,E)\), while it is \emph{equivalence closed} if \(s(C,D) = 1\) is equivalent to \(C \equiv D\).

  \paragraph{Subsumption.} The notion of subsumption can be exploited to obtain an ordering over concepts.
  This, in turn, can be abstracted to the level of a concept similarity measure. How?
  Informally, if \(C\), \(D\) and \(E\) are \elh-concepts and \(C \subsume{K} D \subsume{K} E\), \(D\) is closer to \(E\) than \(C\), thus more similar;
  also, it can be noticed that \(D\) is closer to \(C\) than \(E\).
  These two insights can be captured as \emph{subsumption preserving} and \emph{reverse subsumption preserving}, where the concept similarity measure \(s\) obeys to the constraints \(s(C,E) \le s(D,E)\) (reverse subs.\ preserving) and \(s(C,D) \ge s(C,E)\) (direct subs.\ preserving).

  \paragraph{Structural dependence} is the last property that had been introduced for concept similarity measures.
  The idea behind structural dependence is that as the number of common \emph{features} of two concepts increases and the number of uncommon features is constant, then the similarity of these concepts must be increasing.
  Such distinct features can be thought of as sequence of atoms such that no two atoms in the sequence are in subsumption relation.
  Indeed, a \csm is \emph{structurally dependent} iff, given a sequence \({(C_k)}_{k=1}^n\) of atoms such that if \(1 \le i,j \le n\) and \(i \ne j\), then \(C_i \not\sqsubseteq C_j\), and defining \(D_n := \sqcap_{i=1}^n C_i \sqcap D\) for a given concept name, then \(\lim_{n \to \infty}s(D_n,E_n) = 1\) holds.
