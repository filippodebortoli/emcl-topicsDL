\section{Introduction}

Measuring the similarity of two or more concepts is a task that frequently occurs in many real-world cases, where ontologies are employed, ranging from exploration of knowledge bases for information discovery --- as done in~\cite{GF13} --- to more technical activies, such as ontology alignment, as in~\cite{CHu11}.

As important as a \emph{good} measure for concept similarity is, one has to deal with several issues during the specification, implementation and usage phases of a tool of this kind.
Most of these have been faced in the work done by Lehmann \& Turhan in~\cite{LeTu12}.

Which are the intended properties for a concept similarity measure?
The notion of similarity itself has been largely discussed in the literature (a remarkable reference is the work done in~\cite{Tve77}).
Among the many, possibly different, existing approaches to similarity, the one shows in [] has been followed in this work.
In particular, relevant properties of similarity have been described and formalised; some of these have been drawn from what has been already established in the literature, whereas others have been newly introduced.
These properties are either concerning the mathematical side of a measure or its relation with the underlying representation of the concepts.

Among description logics, a family of formal languages used to represent knowledge in ontologies, the one investigated and chosen as a base for the work herein reportes is the \elh description logic, which has a practical relevance --- for example, in medical ontologies, as in~\cite{GF13} --- and is part of important standards for semantic technologies.
In this setting, it is shown how a similarity measure can be defined, such that it is powerful, yet flexible to edit and efficient to compute, given that one uses the right tools in its construction.

This work also attempts to address another issue: the \emph{portability} and \emph{reliability} of concept similarity measures, across different contexts.
Indeed, many existing measures are domain-related or depend too much on the dataset upon which they are first defined, assessed and evaluated.
The choice, here, is that of defining a framework, a general concept similarity measure which can be specifically instantiated, by means of several parameters; this allows the user to tailor the measure to her particular application, without losing the desired properties, guaranteed to hold by the way the introduce measure \(\simi\) has been specified.

The report has the following structure:
first, some notions concerning \dl{}s are introduced;
then, similarity measure are discussed, with their properties;
next, the framework devised in~\cite{LeTu12} is derived and explained;
finally, some results are shown, reguarding the properties of the new measure.
In the end, an overview of related works is presented, followed by a brief conclusion.
