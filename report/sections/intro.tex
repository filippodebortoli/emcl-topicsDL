\section{Introduction}
\label{sec:intro}

Measuring the similarity of two or more concept descriptions is a task that frequently occurs in several real-world cases.
These applications, where ontologies are employed, range from exploration of knowledge bases for information discovery --- as done in~\cite{GF13} --- to more technical activies, like ontology alignment, as in~\cite{CHu11}.

As important as a \emph{good} measure for concept similarity is, one has to deal with several issues during the specification, implementation and usage phases of a tool of this kind.
Most of these have been faced in the work done by Lehmann and Turhan in~\cite{LeTu12}.

The general notion of \emph{similarity} has been largely treated in the literature --- a remarkable reference is~\cite{Tve77} --- and several publications about similarity of concepts, like~\cite{SemSim} and~\cite{dSF08}, emphasized the need for measures that behave properly in the presence of semantical equivalence of concepts.
However, a thorough analysis made in~\cite{LeTu12} pointed out that many \emph{concept similarity measures}, although claiming to possess such a behaviour, do not preserve \emph{equivalence closure} and \emph{invariance}.

The analysis involved also the definition and formalisation of the desirable properties of a \csm.
Some of these properties, like the mathematical-related or the equivalence-related ones,have been drawn from the literature, whereas others, concerning subsumption and structures of the concepts, have been newly introduced.

Description Logics are a family of formal knowledge representation languages.
The \dl \elh, upon which the \csm \simi has been defined, has a practical relevance: for example, the medical ontology \textsc{snomed ct}~\cite{snomed} employs a variant of \elh, and the Web standard profile \textsc{Owl-el}, described in~\cite{owlEL}, is originated from \elh.
In this setting, it is shown how a \csm can be defined, which achieves \emph{power} --- thanks to \emph{equivalence invariance} and \emph{closure} ---, \emph{flexibility} in the choice of its parameters and \emph{efficiency} in its computation, assuming that the right tools are used in its construction.

Indeed, this work also addresses another issue: the \emph{portability} and \emph{reliability} of concept similarity measures,across different contexts.
Many existing measures are domain-related or depend too much on the dataset upon which they are first defined, assessed and evaluated.
Contrary to this, the framework devised in~\cite{LeTu12} brings into play a \emph{parametric} concept similarity measure: this allows to tailor the measure to a specific scenario.
If the user appropriately instanciates these parameters, the specification of \(\simi\) guarantees that equivalence closure and invariance, along with symmetry and other desirable properties, hold.

This report has the following structure:
first, some notions concerning \dl{}s are introduced;
then, similarity measure are discussed, with their properties;
next, the framework devised in~\cite{LeTu12} is derived and explained;
finally, some results are shown, reguarding the properties of the new measure.
In the end, an overview of related works is presented, followed by a brief conclusion.
