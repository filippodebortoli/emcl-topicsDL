\section{\(\simi{}\): a blueprint for designing CSMs}

  The concept similarity measure \(\simi\), which is going to be introduced in this chapter, is characterised as a \emph{structural measure}: it is defined according to the syntax of the concepts which it is intended to measure, rather than on their interpretations, as in \emph{interpretation-based} measures.

  \paragraph{Preprocessing.}
  The usage of \(\simi{}\) requires a preliminar processing of the involved knowledge base.
  Since only unfoldable TBoxes are considered, it is possible to consider its \emph{expansion}, which is semantically equivalent to the non-expanded version and only contains primitive concept names on the right hand side of concept definitions.
  Then, each concept is replaced by its \elh-normal form; being such a normal form unique up to associativity and commutativity, \emph{equivalence invariance} is ensured for any measure employed to test concept similarity on the \kb.

  \paragraph{Directed and reverse similarity.}
  To test equivalence of two concepts \(C\) and \(D\), a possible way is to check whether subsumption holds in both directions, that is, \(C \sqsubseteq D\) and \(D \sqsubseteq C\).
  This approach can be astracted to the mathematical level, by first defining a \emph{directed similarity} measure --- a generalisation of the subsumption operator ---, to later combine values in both directions by means of another operator.
  The directed similarity measure \(\simi_d\) is going to be derived later in this section; finally, \(\simi\) is going to be defined as \(\simi(C,D) := \simi_d(C,D) \otimes \simi_d(D,C)\).
  The chosen operator should fit certain conditions, which guarantee that the resulting measure complies with the desired properties.
  This is achieved by considering a \emph{fuzzy connector} \(\otimes \colon {[0,1]}^2 \to [0,1]\), which is an operator that fulfills the following conditions, for all \(x, y \in [0,1]\):
  \begin{itemize}
    \item \(x \otimes y = y \otimes x\) --- \emph{commutativity};
    \item \(x \otimes y = 1\) iff \(x = y = 1\) --- \emph{equivalence closure};
    \item if \(x \le y\), then \(1 \otimes x \le 1 \otimes y\) --- \emph{weak monotonicity};
    \item if \(x \otimes y = 0\), then at least one between \(x \ne 0\) or \(y \ne 0\) does not hold.
  \end{itemize}

  \begin{example}[Fuzzy connector]
    The operator \(\otimes \colon {[0,1]}^2 \to [0,1]\) defined as
    \[
    x \otimes y := \frac{x^2y + y^2x}{2} \qquad x,y \in [0,1]
    \]
    is a fuzzy operator.
    The commutativity follows from the commutativity of the sum and the dot product over \([0,1]\). Weak monotonicity holds, because \(x \le y\) implies that \((x \cdot (x+1))/2 \le (y \cdot (y+1))/2\).

    This operator is equivalence closed: assume that \(x \otimes y = 1\) and \(x \ne 1\), \(y \ne 1\), and \(x \le y\) without loss of generality.
    Since \(x^2y = 2 - xy^2\) and \(x^2y \le xy^2\), it follows that \(2-xy^2 \le xy^2\), hence \(xy^2 \ge 1\); however, this is a contradiction to the hypothesis that \(x\), \(y \in [0,1]\), thus at least one between \(x\), \(y\) equals \(1\).
    Suppose that \(y = 1\): then, from \((x^2 + x)/2 = 1\) follows that \(x = 1\); analogously, it can be proved that \(x = 1\) implies \(y = 1\).

    Finally, it is possible to show that \(x = 0\) or \(y = 0\) when \(x \otimes y = 0\).
    Indeed, from
    \[
    \frac{x^2y + y^2x}{2} = 0 \implies x^2y = - y^2x
    \]
    one can conclude that \(x^2y = y^2x = 0\), hence \(x = 0\) or \(y = 0\).
  \end{example}
  In the \(\simi{}\) framework, the choice of a fuzzy operator provides \emph{symmetry} --- thanks to commutativity ---, \emph{equivalence closure} and \emph{sumbsumption preserving}, which follows from weak monotonicity.

  \paragraph{Measuring concept names.}
  The function \(\simi_d\) had been first derived from the function
  \[
  d(C,D) :=
  \frac{\lvert \widehat{C} \cap \widehat{D}\rvert}{\lvert \widehat{C} \rvert},
  \]
  inspired by the Jaccard Index, described in~\cite{Ja01}.
  However, this function does not take into account existential restrictions.
  This issue is partially solved by the following rewriting:
  \[
    \lvert \widehat{C} \cap \widehat{D}\rvert :=
    \sum_{C^\prime \in \widehat{C}}\max_{D^\prime \in \widehat{D}}
    \delta_{C^\prime}^{D^\prime},
  \]
  where \(\delta_{C^\prime}^{D^\prime}\) equals \(1\) when the two concepts are equal, \(0\) otherwise. Later, it will be shown how further modifications, made possible by this one, are going to allow incorporating existential restrictions.
  Another issue arises, with this new definition: it is not possible to distinguish between totally dissimilar and not totally similar concepts, since \(\delta_{C^\prime}^{D^\prime}\) checks syntactical equality of two concepts, assuming that otherwise, they cannot be similar at all.
  Again, the definition is refined by introducing a measure over concept and role name, such that, given adequate conditions, the desired properties for \(\simi\) are respected. A function \(pm \colon \mathbf{C}^2 \cup \mathbf{R}^2 \to [0,1]\) is a \emph{primitive measure} if:
  \begin{enumerate}
    \item for all concept names \(A\), \(B\), \(pm(A,B) = 1\) iff \(A = B\);
    \item for given role names \(r\), \(s\), \(t\):
    \begin{enumerate}
      \item \(pm(r,s) = 1\) iff \(s \sqsubseteq r\),
      \item if \(r \subsume{R} s\), then \(pm(r,s) > 0\),
      \item if \(t \subsume{R} s\), then \(pm(r,s) \le pm(r,t)\).
    \end{enumerate}
  \end{enumerate}
  This definition is sufficient to provide \emph{equivalence closure} and \emph{subsumption preserving} in \(\simi\).

  To be able to deal with existential restrictions, all the possible cases should be covered:
  \begin{enumerate}
    \item When similarity between concept names is computed, the primitive measure \(pm\) is sufficient, since it handles this case directly.
    \item If a concept name and an existential restriction are compared, the similarity will be \(0\).
    \item In the final case, it would be needed two compute the similarity of two existential restrictions, namely \(\exists{}r.C^\star{}\) and \(\exists{}s.D^\star{}\).
    To achieve this, both role names and concept names appearing should be considered.
    While the roles are handled by \(pm\), a recursive call of the measure over \(C^\star\), \(D^\star\) is needed.
    To combine these two values, and to establish how much influence does each one exert on the final measure, an affine combination is introduced, along with a value \(w \in (0,1)\):
    \[
    d(\exists{}r.C^\star,\exists{}s.D^\star) :=
    pm(r,s) \cdot \left(w + (1-w)d(C^\star,D^\star)\right).
    \]
  \end{enumerate}
  In this way, the case \(d(C^\star,D^\star) = 0\) allows one to understand whether the roles are similar --- then the similarity equals \(w\), otherwise it would be equal to \(0\).
  What is a \emph{good} value for \(w\)? By default, \(w = 0.01\).
  A feasible value for \(w\) may be proportionally decreasing with respect to the natural number \(n\) such that the two concepts \(C\), \(D\) of the form
  \[
  C = \overbrace{\exists{}r.\dotsb{}\exists{}r.}^{n\;\text{times}}A \quad
  D = \overbrace{\exists{}r.\dotsb{}\exists{}r.}^{n\;\text{times}}B
  \]
  with \(pm(A,B) = 0\) are almost, if not totally similar.

  \paragraph{Information on similarity.}
  The last issue that was dealt with in~\cite{LeTu12} concerned the \(\max{}\) operator appearing in the definition of the similarity measure developed so far.
  Indeed, the claim is that such an operator is not making full use of the available information, to compute the similarity of two concepts; that happens, because only the \emph{best matching} atom in the set \(\widehat{D}\) is kept, discarding all the others, which might still be similar to the considered atom of \(\widehat{C}\).

  The maximum operator fits into a class of functions called \emph{triangular conorms}; %, presented in \todo{Insert reference here.[14] on paper.};
  in particular, it is an instance of a \emph{bounded \(t\)-conorm}.
  \begin{definition}%[Bounded t-conorm]
    An operator \(\oplus \colon {[0,1]}^2 \to [0,1]\) is a \emph{bounded t-conorm} if it satisfies the following conditions: \(\oplus\) is commutative, associative, \(0\) is the neutral element and for all \(x\), \(y\), \(z\), \(w \in [0,1]\),
    \begin{itemize}
      \item if \(x \le z\) and \(y \le w\), then \(x \oplus y \le z \oplus w\), i.e. \(\oplus{}\) is \emph{monotonic};
      \item if \(x \oplus y = 1\), then \(x = 1\) or \(y = 1\), i.e. \(\oplus{}\) is \emph{bounded}.
    \end{itemize}
  \end{definition}
  Moreover, \(\max\) is the strongest \(t\)-conorm, which means that all the other \(t\)--conorms yield a value equal or greater than \(\max\); this, intuitively, reflects the fact that \(\max\) does not make good use of the available information.
  Finally, \(0\) is the unit element of \(t\)-conorms: this implies that totally dissimilar concepts are not influencing the final value of the measure.
  Thus, in the final definition of \(\simi_d\), \(\max\) will be replaced by a \emph{bounded \(t\)-conorm}, which is a parameter of the framework.

  \paragraph{Relevance of concept names.}
  Up until now, all the atoms of \(\mathcal{C}\)(\elh) had the same relevance, in determining how concepts differ from each other.
  However, it is likely that some concepts might be prioritised in some use cases.
  Hence, another parameter that has been added to the framework is a function that specifies how much an atom influences the similarity between concepts.
  \begin{definition}%[Weighting function]
    A \emph{weighting function} is a mapping \(g \colon \mathbf{A} \to \mathbb{R}_+\), where \(\mathbf{A}\) is the set of \emph{atoms}.
  \end{definition}

  Since \(\mathbf{A}\) can contain infinitely many elements, it would be desirable to find a way to define a weighting function by using only some "essential" information.

  \paragraph{Primitive weighting.}
  Thanks to the preprocessing phase, each concept in \elh-normal form does not contain any instance of \(\top\): hence, it is not needed to weigh it.
  Another aspect of this normal form is that it is possible to weigh just a subset of \(\mathbf{R}\), that is the image of the function \(ch\) used to compute the \elh-normal form; in this way, at most \(\lvert \mathbf{R} \rvert\) roles would need to be weighted.
  In this case, all the role names are used.
  Finally, concept names can be either primitive or defined:
  in the latter case, thanks to preprocessing again, a unique definition can be find in the expanded TBox, which right side contains exclusively primitive concept names.

  Using these facts, the following definition is introduced.
  \begin{definition}[Primitive weighting function]
    A \emph{primitive weighting function} is a mapping \(f \colon \mathbf{P} \cup \mathbf{R} \to \mathbb{R}_+\), where \(\mathbf{P}\) is the subset of \(\mathbf{C}\) of \emph{primitive} concept names.
  \end{definition}
  Given a weighting primitive function, it is possible to extend it to a weighting function \(g\) in the following way:
  \begin{equation}\label{prim-weigh}
    g(C) :=
    \begin{cases}
      f(C) & C \in \mathbf{P} \\
      f(r) & C = \exists{}r.D, D \in \mathcal{C}(\mathcal{ELH}) \\
      \max_{D^\prime \in \widehat{D}}f(D) & C \equiv D \; \text{appears in the expanded TBox.}
    \end{cases}
  \end{equation}
  \begin{proposition}
    With the expansion given in~\eqref{prim-weigh}, every primitive weighting function uniquely determines a weighting function.
  \end{proposition}
  \begin{proof}
    Assume that the weighting functions \(g\), \(h\) are both derived from \(f\) as in~\eqref{prim-weigh}.
    \begin{enumerate}
      \item If \(C \in \mathbf{P}\), then
      \(g(C) = f(C) = h(C)\) by definition;
      \item if \(C =  \exists{}r.D, D \in \mathcal{C}(\mathcal{ELH})\), then \(g(C) = f(r) = h(C)\);
      \item if \(C \equiv D\) is the \emph{unique} definition of \(C\) in the TBox \(\mathcal{T}\), then
      \(g(C) = \max_{D^\prime \in \widehat{D}}f(D) = h(C)\).
    \end{enumerate}
    Thus, it is possible to conclude that \(g(C) = h(C)\) for all \(C \in \mathbf{A} \setminus \lbrace \top \rbrace\).
  \end{proof}
  This approach can be further refined, for example, by substituting the \(\max\) function with another, perhaps more suitable function.
  %Notice that, given a primitive weighting function, this extension to a weighting function can be computed in polynomial time (this last claim needs to be verified).

\subsection{The measure of directed similarity: \(\simi_d\)}

  \begin{definition}[\(\simi_d\)]\label{simi-d}
    The measure of \emph{directed similarity} \(\simi_d \colon {\mathcal{C}(\mathcal{ELH})}^2 \to [0,1]\), given the following parameters:
    \begin{enumerate}
      \item a bounded \(t\)-conorm \(\oplus\),
      \item a primitive measure \(pm\),
      \item a weighting function \(g\),
      \item \(w \in (0,1)\),
    \end{enumerate}
    is defined as:
    \begin{align}
      \simi_d(\top,D) &= 1 \quad
      \text{for all}\; D \in \mathcal{C}(\mathcal{ELH}) \\
      \simi_d(C,D) &=
      \frac{\sum_{C^\prime \in \widehat{C}}\left(g(C^\prime) \cdot \bigoplus_{D^\prime \in \widehat{D}} \simi_d(C^\prime, D^\prime)\right)}%
      {\sum_{C^\prime \in \widehat{C}}g(C^\prime)} \quad \text{if}\; C \ne \top \label{simi:general}\\
      \simi_d(A,B) &= pm(A,B) \quad \text{for atoms}\; A,B \\
      \simi_d(\exists{}r.C,\exists{}s.D) &=
      pm(r,s) \cdot \left(w + (1-w)\simi_d(C,D)\right) \\
      \simi_d(C,D) &= 0 \quad \text{otherwise}.
    \end{align}
  \end{definition}

  \begin{definition}[\(\simi\)]
    The similarity measure \(\simi\) is defined as \(\simi(C,D) := \simi_d(C,D) \otimes \simi_d(D,C)\), where \(\otimes{}\) is a fuzzy connector.
  \end{definition}

  The following lemma shows an interesting property of \(\simi_d\), which can be used to prove some of its important features.

  \begin{lemma}
    Let \(C\), \(D\) and \(E\) be \elh-concept descriptions.
    Then,
    \begin{enumerate}
      \item \(\simi_d(C,D) = 1\) iff \(D \sqsubseteq C\);
      \item if \(D \sqsubseteq E\), then \(\simi_d(C,E) \le \simi_d(C,D)\).
    \end{enumerate}
  \end{lemma}
  \begin{proof}
    The proof of this theorem can be found in~\cite{LeTu12}.
  \end{proof}
  \begin{theorem}
    The \csm \(\simi\) is symmetric, closed and invariant under equivalence and preserves subsumption, for each instantiation of its parameters.
  \end{theorem}
  \begin{proof}
    Let \(C\), \(D\) and \(E\) be \elh-concept descriptions.

    The symmetry of \(\simi\) follows from the commutativity of the fuzzy connector.

    The invariance under equivalence of \(\simi\) is guaranteed by the preprocessing step.
    Indeed, two semantically equivalent \elh-concepts are transformed into the same \elh-normal form.
    By assuming that \(C \equiv D\), this insight allows the conclusion \(\simi(C,E) = \simi(D,E)\).

    The closure under equivalence is proved by the following equivalence chain:
    \begin{equation*}
      \begin{split}
        C \equiv D &\iff C \sqsubseteq D \land D \sqsubseteq C \iff
        \simi_d(C,D) = 1 \land \simi_d(D,C) = 1 \\ &\iff
        \simi(C,D) = \simi_d(C,D) \otimes \simi_d(D,C) = 1.
      \end{split}
    \end{equation*}

    In a similar way, subsumption preserving can be proved.
  \end{proof}
  The \csm \(\simi\) fulfills most of the desired properties.
  Indeed, in~\cite{LeTu12}, it is shown that it is \emph{equivalence closed}, \emph{equivalence invariant}, \emph{symmetric} and \emph{subsumption preserving} for every choice of the parameters.
  For adequates choice of the fuzzy connector and the weighting function, \emph{structural dependence} can be achieved.
  The only properties that cannot be guaranteed by \(\simi\) are \emph{triangle inequality} and \emph{reverse subsumption preserving}, since similarity values between the atoms of the involved concepts are not taken into account, in the computation of the similarity value of the concepts.

  \begin{proposition}
    The \csm \(\simi\) can be computed efficiently: if the provided fuzzy connector, bounded \(t\)--conorm, primitive measure and weighting function can be computed in polytime, so can \(\simi\), with time polynomial in the size of the measured concepts.
  \end{proposition}
  \begin{proof}
    From the assumption that the fuzzy connector can be computed in polytime, it suffices to show that \(\simi_d\) can be computed in polytime.
    The assumption that the primitive measure can be computed in polytime allows to conclude that all the cases of \(\simi\) where \emph{atoms} are measured run in polynomial time.

    In the most general case~\eqref{simi:general}, the number of similarity values that have to be computed amounts to \(\lvert \widehat{C} \rvert \cdot \lvert \widehat{D} \rvert\).
    Moreover, the depth of the recursion tree is bounded by the size of the concepts \(C\) and \(D\).
    Thus, we can conclude that \(\simi\) runs in polynomial time, in the size of the measured concepts.
  \end{proof}
