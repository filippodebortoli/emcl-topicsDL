\section{Related work}

  \paragraph{The triangle inequality.} While the majority of similarity measures actually defined do not respect the triangle inequality, for \el-concept descriptions an alternative approach had been developed in~\cite{DAB14}, where concepts in \el normal form are translated into \emph{description trees}; then, the similarity measures in there defined, based on the description trees of the concepts, though being non monotone and not structurally dependent, respect triangle inequality.

  \paragraph{Subsumption preservation.} An exploit of subsumption preservation as a guiding principle of similarity measures had been adopted in~\cite{Se16},
  as a way to improve evaluation methods, in which subsumption preservation is used to populate query inventories, which are then used to perform the evaluation of word embeddings.

  \paragraph{General TBoxes.} A general treatment of TBoxes in the context of similarity evaluation of relaxed instances of \el-concepts, had been included in a follow-up work, which introduces a treatment for both the general and the unfoldable case (\cite{Ec14}).

  \paragraph{Approximation.} Approximation grades, induced by concept similarity measures and complexity of relative threshold \dl{}s, are discussed in~\cite{Ba17}, where it is shown how \csm{}s for \el description logics can be employed to derive a grade membership function and consequently a threshold \dl.

  \section{Conclusion}

  In this report, the concept similarity measure \(\simi\), which possess some interesting and useful properties, has been introduced.

  Some of its properties have been explored, such as invariance and closure under equivalence or symmetry, and it has been shown how its computation can be considered efficient.

  The path towards the definition of such a measure has been illustrated, with a focus on each single parameter, along with the motivations that justified their usage.
  Most of the desired properties have been successfully incorporated, while others, like triangle inequality, have been not included, but may be possibly achieved.

  While \elh is a good and widely used \dl, the question is whether a similar measure might be extended to more expressive description logics, without losing its peculiar features.
  It would also be worthy to investigate the possibility of using general TBoxes, where cycles may break the behaviour of \(\simi\) as it is defined now.

  Finally, it would be interesting to obtain a full implementation of \(\simi\), in order to be able to assess it against other existing measures, as well as to start using it to evaluate concept similarity in ontologies.
