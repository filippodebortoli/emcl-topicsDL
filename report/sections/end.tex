\section{Related work}

\paragraph{Improving performances.}
Another \csm for \elh has been investigated in~\cite{simEL}, that outperforms \(\simi\), while maintaining the same similarity properties.
The approach adopted by the authors involves the usage of the so-called \emph{\elh-concept description trees} as a preprocessing step, and its motivated by the homomorphism-based structural subsumption characterisation.
Despite looking similar to \(\simi\) in its definition, this \csm takes a bottom-up approach, which allows for rejection of exceeding recursive calls and the reuse of partial solutions to subproblems.
\(\simi\), with its top-down approach, can potentially have an increasing amount of repeated subroutines which are avoided by this new measure.

\paragraph{The triangle inequality.}
Among the desired properties for a \csm, triangle inequality is the least achieved.
Can this be obtained, without sacrificing other, more relevant properties?
In~\cite{TriEq}, the translation of \el-concepts to description trees is also employed. Then, a dissimilarity measure based on \emph{concept relaxations} is defined. A \emph{concept relaxation} can be intuitively understood as a stepwise generalization of a concept.
This new dissimilarity measure respects the triangle inequality and fulfills all the other desired properties, except for structural dependency.

  \paragraph{General TBoxes.} A general treatment of TBoxes in the context of similarity evaluation of relaxed instances of \el-concepts, had been included in a follow-up work, which introduces a treatment for both the general and the unfoldable case (\cite{Ec14}).

  \paragraph{More expressive \dl{}s.}
  Extensions to \simi which investigate the tradeoff between adding measuring capabilities and losing  features of similarity measures have been discussed in~\cite{Lae-dip}, where properties of \gls{csm} have also been inspected using methods from the field of Formal Concept Analysis.

  \paragraph{Approximation.} Approximation grades, induced by concept similarity measures and complexity of relative threshold \dl{}s, are discussed in~\cite{Ba17}, where it is shown how \csm{}s for \el description logics can be employed to derive a grade membership function and consequently a threshold \dl.

  \section{Conclusion}

  In this report, the concept similarity measure \(\simi\), which possesses some interesting and useful properties, has been introduced.

  Some of its properties have been explored, such as invariance and closure under equivalence or symmetry, and it has been shown how its computation can be considered efficient.

  The path towards the definition of such a measure has been illustrated, with a focus on each single parameter, along with the motivations that justified their usage.
  Most of the desired properties have been successfully incorporated, while others, like triangle inequality, have been not included, but may be possibly achieved.

  Finally, it would be interesting to obtain a full implementation of \(\simi\), in order to be able to assess it against other existing measures, as well as to start using it to evaluate concept similarity in ontologies.